% Tiivistelmät tehdään viimeiseksi. 
%
% Tiivistelmä kirjoitetaan käytetyllä kielellä (JOKO suomi TAI ruotsi)
% ja HALUTESSASI myös samansisältöisenä englanniksi.
%
% Avainsanojen lista pitää merkitä main.tex-tiedoston kohtaan \KEYWORDS.

\begin{fiabstract}

Ohjelmistoalan kilpailutilanteen kiristyessä yritykset etsivät jatkuvasti tapoja
tehostaa toimintaansa. Ketterien menetelmien on todettu parantavan tehokkuutta
sekä laatua, mikä nostaa ne houkuttelevaksi vaihtoehdoksi perinteisille
kehitysmenetelmille. Ketterien menetelmien käyttöönotto on kuitenkin haastavaa
suurissa yrityksissä, sillä ne on alun perin suunniteltu sovellettavaksi
pienissä projekteissa.

Tämän työn tavoitteena on selvittää mitkä tekijät vaikuttavat ketterän
kehitysmallin organisaatiomuutoksen läpiviemiseen suuressa organisaatiossa,
miten muutokset yleensä toteutetaan sekä miksi muutokseen ryhdytään. Hyödyntäen
järjestelmällisen kirjallisuustutkimuksen menetelmää löydettiin 30 ensisijaista
tutkimusta, jotka antoivat vastauksia näihin kysymyksiin. Tuloksissa esitellään
tyypillisiä muutoksen toteutustapoja, haasteita ja menestyksen tekijöitä.

Organisaatiomuutokseen ryhdyttiin kolmesta pääasiallisesta syystä, joita olivat
yleinen tarve tehostaa toimintaa, tiedostettujen prosessiongelmien poistaminen
ja tarve vastata markkinoiden muutoksiin nopeammin. Organisaatiomuutoksen
menestyksen tai epäonnistumisen keskeisimmäksi tekijäksi nousi tapa, jolla
muutosta johdettiin. Määrätietoinen johtaminen oli keskeisin menestyksen tekijä,
ja keskeisimmät ongelmat johtuivat vaikeuksista muodostaa yhtenäistä suuntaa
muutokselle kautta organisaation. Muita tärkeitä menestyksen tekijöitä oli
riittävä koulutuksen järjestäminen sekä yhteisöllisyyden luominen.
Pilotointi ja ulkopuolisten konsulttien käyttö oli tyypillistä muutoksissa.


%Tiivistelmän tyypillinen rakenne: 
%(1) aihe, tavoite ja rajaus 
%(heti alkuun, selkeästi ja napakasti, ei johdattelua);
%(2) aineisto ja menetelmät (erittäin lyhyesti);
%(3) tulokset (tälle enemmän painoarvoa); 
%(4) johtopäätökset (tälle enemmän painoarvoa).

%
%Tiivistelmätekstiä tähän (\languagename). Huomaa, että tiivistelmä tehdään %vasta kun koko työ on muuten kirjoitettu.
\end{fiabstract}

%\begin{svabstract}
%  Ett abstrakt hit 
%%(\languagename)
%\end{svabstract}

%\begin{enabstract}
% Here goes the abstract 
%%(\languagename)
%\end{enabstract}
