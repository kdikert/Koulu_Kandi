% -------------- Symbolit ja lyhenteet --------------


\addcontentsline{toc}{section}{Käytetyt symbolit ja lyhenteet}

\section*{Käytetyt lyhenteet ja termit}
%\section*{Abbreviations and Acronyms}

\begin{center}
\begin{tabular}{p{0.2\textwidth}p{0.65\textwidth}}
CI  & Continuous Integration; Jatkuva yhdentäminen, eli menettely jossa
      lähdekoodiin tehdyt muutokset liitetään osaksi jaettua kehityshaaraa
      mahdollisimman usein. \\ 
XP  & Extreme Programming; Ketterä ohjelmistokehitysmenetelmä. \\
CCB & Change Control Board; Muutoksenhallintalautakunta on organisaation elin
      joka tarkkailee ja hyväksyy suunnitteluun tehtäviä muutoksia.
JAD & Joint Application Development; Ketterä ohjelmistokehitysmenetelmä. \\
DSDM & Dynamic Systems Development Method; Ketterä ohjelmistokehitysmenetelmä.\\
\end{tabular}
\end{center}

\vspace{10mm}

Tähän voidaan listata kaikki työssä käytetyt lyhenteet. Lyhenteistä
annetaan selityksenä sekä alkukielinen termi kokonaisuudessaan
(esim. englanninkielinen lyhenne avattuna sanoiksi) että sama
suomeksi. Jos suoraa käännöstä ei ole tai sellaisesta on vaikea saada
sujuvaa, voi käännöksen sijaan antaa selityksen siitä, mitä kyseinen
käsite tarkoittaa. Jos lyhenteitä ei esiinny työssä paljon, ei tätä
osiota tarvita ollenkaan. Yleensä luettelo tehdään, kun lyhenteitä on
10--20 tai enemmän. Vaikka lyhenteet annettaisiinkin tässä
keskitetysti, ne pitää silti avata sekä suomeksi että alkukielellä
myös itse tekstissä, kun ne esiintyvät siellä ensi kertaa.  Käytetyt
lyhenteet -osion voi nimetä myös ``Käytetyt lyhenteet ja termit'', jos
luettelossa on sekä lyhenteitä että muuta käsitteenmäärittelyä.

\textbf{TIK.kand suositus: Lisää lyhenne- tai symbolisivu, kun se
  näyttää luontevalta ja järkevältä. (Käytä vasta kun lyhenteitä yli 10.)}


