% --------------------------------------------------------------------

\section{Johdanto}

--> yritykset etsivät jatkuvasti tapoja parantaa toimintaansa (kiristyvässä
kilpailutilanteessa). --> Tämä on nostanut ketterät ohjelmistomenetelmät
mahdolliseksi ratkaisuksi toiminnan tehostamisessa.

--> Myös lean (Poppendick) on mahdollista --> miten se suhteutuu agileen

--> Tämän työn tavoitteenna on esittää semisystemaattisen kirjallisuuskatsauksen
tulokset -->


Vanhoja toimintamalleja:
-- waterfall
-- RUP
-- CMMi


You searched for: agile team communication
* Analyzing the modes of communication in agile practices; Bhalerao, S.; Ingle, M.
* Usage and Perceptions of Agile Software Development in an Industrial Context: An Exploratory Study; Begel, A.; Nagappan, N.
* Scaling Agile: Finding your Agile Tribe
* Introducing Agile Development Practices from the Middle; Thomas, J.

\subsection{Työn rakenne (jätä pois yksinäinen otsikko\ldots)}

Tämä työ on jaoteltu seuraavasti:
Seuraavassa luvussa esittelen tutkimuksent taustan ja määrittelen työn
tavoitteet. Luvussa \ref{sec:menetelma} esittelen semisystemaattisen
kirjallisuuskatsauksen menetelmän jota olen käyttänyt tutkimuksen
suorittamiseen. Luku \ref{sec:tulokset} esittelee kirjallisuuskatsauksen
tulokset. Lopuksi esittelen tuloksista tehdyt johtopäätökset sekä teen
yhteenvedon työstä.


% --------------------------------------------------------------------

\section{Työn taustat ja tavoitteet}
\label{sec:tausta}

--> mitä ovat ketterät menetelmät


\subsection{Aikaisemmat tutkimukset}

--> Ei taida olla aikaisempia laajempia katsauksia
    * Miten perustellaan??
    
--> 

\subsection{Tutkimuskysymykset}

--> RQ1
 * RQ1.1
 * RQ1.2
 * RQ1.3

\subsection{Työn rajaus}

--> Vain isot organisaatiot
--> Mikä on iso organisaatio? --> Vapaasit määriteltynä: sellainen jossa voidaan
olettaa että perinetisen projektisuunnittelun korvaaminen ketterillä
menetelmillä voi aiheuttaa koordinaatiovaikeuksia.

--> Organisaatiossa on oltava olemassa oleva toimintamalli

% --------------------------------------------------------------------

\section{Tutkimusmenetelmä}
\label{sec:menetelma}

\citep{refworks:148}

--> Esihaut: IEEE, hakusanat --> Löytyi muutamia jotka eivät ole mukana muuten?

--> Näkökulmat hakuun, vastaavat hakusanat ja boolean-lause: agile,
transformation, large scale

--> Käytetyt hakukoneet

--> Viitteiden validiteetti (kuinka suuri luottamus)

--> Seulontamenetelmä: Otsikko ja abstrakti --> Kokoteksti

--> ''Data extraction form''

% --------------------------------------------------------------------

\section{Tulokset}
\label{sec:tulokset}

--> Millä aikavälillä tutkimuksia on tehty

% --------------------------------------------------------------------

\section{Johtopäätökset}
\label{sec:johtopaatokset}




% --------------------------------------------------------------------

\section{Yhteenveto}
\label{sec:yhteenveto}




