% --------------------------------------------------------------------

\section{Johdanto}

Tämä dokumentti kertoo sekä TIK.kand-kandidaattiseminaarikurssin
ohjeistuksesta kandidaatintyön muotoseikoista että yleisemmin
\LaTeX{}-järjestelmän käyttämisestä opinnäytetyön kirjoittamisessa.

\subsection{Tämän dokumentin tausta}

Lisäksi suomen kielen lehtori Sanni
Heinzmann on kirjoittanut rakenteellisia vinkkejä luvuittain
(tiivistelmä, käytetyt lyhenteet, johdanto, loppuluku, liitteet).
Viimeisin päivitys on syyskuulta 2011.

\subsection{Johdantoluku}

Työn ensimmäinen luku on aina johdanto. Kandidaatintyön laajuudessa
sitä ei ole tarvetta jakaa alalukuihin, diplomityössä ja muissa
isommissa töissä sekä tutkimusraporteissa alaluvut ovat mahdollisia
(esim. 1.1 Tutkimusongelma, 1.2 Aineisto ja tutkimusmenetelmä, jne.).

Käsittele nämä aiheet johdannossa (jotakuinkin tässä järjestyksessä):
%
\begin{itemize}
 \item Johdatus aihepiiriin 
(ei liian laajasti, vaan relevantisti ja napakasti)
%
 \item Tutkimuskohteen esittely (MITÄ tämä työ tutkii? 
Kerro työstä/tutkimusaiheesta, ei omasta kirjoitusprosessistasi 
tai omasta kiinnostuksestasi.)
%
 \item Tutkimuksen perustelu: ongelma tai aukko 
(aiemmassa tutkimuksessa on aukko, tai siitä nousee esiin 
kysymys, johon tässä etsitään vastausta)
%
 \item Tutkimusongelma / -kysymykset (koko työsi sydän, 
jonka pitäisi näkyä ''punaisena lankana'' koko työn läpi)
 \item Tavoitteet (Käytä konkreettisesti sanaa ''tavoite'')
 \item Rajaus (Mitä tämä työ EI tutki)
 \item Menetelmä, aineisto, teoreettinen kehys (Esittele, 
MITEN em. aihetta tutkitaan)
 \item Tulokset? (Johdannossa on ihan hyvä antaa lyhyesti 
tietoa päätuloksista, mutta ei pakko)
 \item Työn sisältö ja rakenne (Esittele, miten työn punainen 
lanka etenee, viittaukset työhön: 
``ensin, sitten, seuraavaksi, luvussa 3'' jne.)
\end{itemize}

% --------------------------------------------------------------------

\section{Kandidaatintyön rakenne- ja muotoseikat}
\label{sec:esimluku}

Tässä luvussa esitellään kandidaatintyön muotovaatimuksia
tällä kurssilla. Muutamat alkuperäiset lähteet ovat saattaneet
kadota organisaatio- ja tietojärjestelmämuutoksissa, kun
Into-järjestelmä on korvannut WWW-sivustoja.

\subsection{TKK:n kandidaattityöryhmän ohjeistus}

Yleiset kandidaatintyön muotovaatimukset on annettu TKK:n
kandidaattityöryhmän päätöksellä 14.11.2006 ja ne ovat
kokonaisuudessaan saatavissa osoitteessa
\url{http://www.tkk.fi/fi/opinnot/opintohallinto/paatokset/kandi20061114.pdf}.
Tässä luvussa annetaan lyhyt, selvennetty ja joiltakin osiltaan
karsittu yhteenveto kyseisistä ohjeista. Seuraavassa viitataan siis
edellä mainittuihin TKK:n kandidaatintyön ohjeisiin (esim. ``luku 3''
tarkoittaa TKK:n kandidaatintyön ohjeiden lukua kolme).

\textbf{TIK.kand suositus: Lue alkuperäiset ohjeet erityisesti
  silloin, jos et kirjoita työtä annettua \LaTeX{}-pohjaa käyttäen.}

TKK:n kandidaatintyön ohjeissa käsitellään työn rakennetta (luvussa 3)
ja muotoseikkoja (luvussa 4). Yleisesti todetaan kandidaatintyöstä
seuraavaa:
%
\begin{quotation}
\noindent \it
Kandidaatintyö voi perustua teoreettisen taustan tarkasteluun 
ja sen analysointiin sekä johtopäätösten tekoon tai kokeelliseen osioon ja 
tulosten analysointiin sekä johtopäätösten tekoon
tai edellisten yhdistelmään.
Kandidaatintyön rakenteen tulee olla hyvän tieteellisen kirjoittamisen 
käytännön mukainen
ja sisältävän vähintään seuraavat osat: [$\ldots$] (Luku 3)
\end{quotation}

Rakenteen osia ovat: nimiölehti, tiivistelmä, sisällysluettelo,
symboli- ja lyhenneluettelo (työn luonteen vaatiessa voi puuttua),
johdanto, aikaisempi tutkimus (työn luonteen vaatiessa teoreettinen
tau

% --------------------------------------------------------------------

