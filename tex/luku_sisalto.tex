% --------------------------------------------------------------------

\section{Johdanto}

--> yritykset etsivät jatkuvasti tapoja parantaa toimintaansa (kiristyvässä
kilpailutilanteessa). --> Tämä on nostanut ketterät ohjelmistomenetelmät
mahdolliseksi ratkaisuksi toiminnan tehostamisessa.

--> Ketterien ohjelmistokehitysmenetelmien on kuitenkin väitetty olevan
soveltumattomia suurille yrityksille.

--> Agiili on tehokkaampaa
  * Empirical Studies on Quality in Agile Practices: A Systematic Literature Review; Dyba 2009

--> Viittaus edellisten töiden puutteeseen??

--> Tämän työn tavoitte on esittää nykyinen tutkimuksen tila ketterien
ohjelmistokehitysmenetelmien käyttöönotosta suurissa organisaatioissa. Tämä työ
on toteutettu mukaillen jäejestelmällisen kirjallisuuskatsauksen muotoa, jotta
lopputuloksena olisi mahdollisimman laaja ja puolueeton katsaus tutkimusalan
nykytilaan.

Kirjallisuustutkimuksen tuloksena havaittiin että \ldots (yhteenveto tuloksista)
(Yhteenveto johtopäätöksistä -- ei ollut suositeltu)


\subsection{Työn rakenne (jätä pois yksinäinen otsikko\ldots)}

Tämä työ on jaoteltu seuraavasti:
Seuraavassa luvussa esittelen tutkimuksen taustan ja määrittelen työn
tavoitteet. Luvussa \ref{sec:menetelma} esittelen semisystemaattisen
kirjallisuuskatsauksen menetelmän jota olen käyttänyt tutkimuksen
suorittamiseen. Luku \ref{sec:tulokset} esittelee kirjallisuuskatsauksen
tulokset. Lopuksi esittelen tuloksista tehdyt johtopäätökset sekä teen
yhteenvedon työstä.


% --------------------------------------------------------------------

\section{Työn taustat ja tavoitteet}
\label{sec:tausta}

\subsection{Perinteiset ohjelmistokenityksen menetelmät}

Vanhoja toimintamalleja:
-- waterfall
-- RUP
-- CMMi

--> Mitä näistä voi sanoa? --> Onko niitä edelleen käytössä?

--> Yhteenveto ongelmista?

--> Ihmisläheinen ''kaoottinen'' hallinta on välttämätöntä ohjelmistokehityksessä.
Suunnitalmavetoinen ja hallintaan perustuva ohjaus ei voi toimia, sillä
ohjelmistokehitys on aina lähempänä uuden tuotteen suunnittelua kuin
tehdaslinjastoa vastaavaa toistuvaa prosessia. <Schwaber \& Beedle 2002>


\subsection{Ketterät ohjelmistokenityksen menetelmät}

--> mitä ovat ketterät menetelmät

Verrattavia toimintamalleja:
--> Myös lean (Poppendick) on mahdollista
--> miten se suhteutuu agileen


--> Ketterät kehitysmallit pyrkivät tuomaan parannuksia perinteisiin
ohjelmistokehityksen menetelmiin. --> Alla listattu mitä parannuksia voi olla

-- Kommunikaation parantaminen
  * Dokumentaation vähentäminen
  * Parempi yhteys asiakkaaseen
  * Kts. Bhalerao

-- Nopeampien muutosten salliminen
  * XP, Beck: Embracing change
  * Nopeammat syklit <Poole>

==> Mukautuimskyky?

-- Virheiden vähentäminen
  * Virheprosessin keventäminen (Korhonen??) -- ja edut?
  * Kts. Ileva
  * Kts.: Empirical Studies on Quality in Agile Practices: A Systematic Literature Review; Sfetsos

--> Yhteenveto ongelmista ketteryyden soveltamiseen ??
  * Lack of control? --> Pelko??


\subsection{Aikaisemmat tutkimukset}

--> Ei taida olla aikaisempia laajempia katsauksia
    * Miten perustellaan??
    * Miten tarkastetaan edellinen tutkimus?
    
--> 

\subsection{Tutkimuskysymykset}

Tämän työn tutkimuskysymyksenä on: \textit{Mitkä tekijät vaikuttavat ketterän
kehitysmallin organisaatiomuutoksen läpiviemiseen isossa organisaatiossa?}
Tutkimuskysymys on edelleen jaettu kolmeen alla listattuun alakysymykseen.

\begin{itemize}
\item Miksi organisaatiomuutokseen ryhdytään?
\item Minkälaisia organisaatiomuutoksia on raportoitu?
\item Mitkä ovat menestyksen ja ongelmien tekijät?
\end{itemize}

\subsection{Työn rajaus}

--> Vain isot organisaatiot
--> Mikä on iso organisaatio? --> Vapaasit määriteltynä: sellainen jossa voidaan
olettaa että perinetisen projektisuunnittelun korvaaminen ketterillä
menetelmillä voi aiheuttaa koordinaatiovaikeuksia.

--> Organisaatiossa on oltava olemassa oleva toimintamalli

% --------------------------------------------------------------------

\section{Tutkimusmenetelmä}
\label{sec:menetelma}

Tässä työssä käytetty tutkimusmenetelmä myötäilee Kitchenhamin esittämää mallia
ohjelmistotuotannon alan kirjallisuustutkimukseen. Valitun menetelmän tavoite on
muodostaa yhtenäinen kuva tämänhetkisestä tutkimuksesta työn aihealueella.
Kirjallisuustutkimuksen suorittaminen sisältää viisi pääasiallista vaihetta:
hakujen suunnittelu, aineiston seulonta, aineiston laadun arviointi, tiedon
poiminta sekä synteesin tekeminen. \citep{Kitchenham2007}

Tässä luvussa esittelen miten olen soveltanut kirjallisuustutkimuksen menetelmän
neljää ensimmäistä vaihetta. Luvussa \ref{sec:tulokset} esittelen tutkimusen
synteesin tulokset. Järjestelmällisen kirjallisuustutkimukseen kuuluu lisäksi
ulkopuolinen katselmointi ja systemaattinen tutkimusprotokollan seuraaminen
\citep{Kitchenham2007}, mutta niiden soveltaminen ylittää tämän työn laajuuden.

\subsection{Hakujen suunnittelu}
Hakujen suunnittelulla tarkoitetaan käytettävien elektronisten aineistojen
valintaa ja hakulausekkeiden muodostamista. Elektronisina aineistoina käytin
seuraavia ohjelmistotuotannon alan julkaisuja hyvin kattavia tietokantoja:
IEEExplore <link>, ACM <link>, Scopus (ScienceDirect??) <link>, ProQuest <link>.
Näiden lisäksi suoritin haun <XP~Conference> arkistoon.

Hakulausekkeet ovat boolen logiikalla muodostettavia lausekkeita, jotka
määräävät avainsanat joiden tulee esiintyä hakutuloksissa. Hakulusekkeita varten
määrittellään joukko näkökulmia jotka edustavat tutkimuskysymysten aihepiiriä.
Jokaista näkökulmaa kohden etsitään yleisimmin käytettyjä avainsanoja sekä
niiden synonyymejä. Lopuksi hakulausekkeet muunnetaan boolen lausekkeiksi
käyttämällä \texttt{OR}-operattoria avainsanojen välillä, sekä
\texttt{AND}-operaattoria näkökulmien välillä.
Taulukko~\ref{table:hakulausekkeet} esittää tässä työssä käytetyt näkökulmat ja
niitä vastaavat avainsanat.

\begin{table}
    \begin{tabular}{|l|l|}
        \hline
        Näkökulma           & Avainsanat   \\ \hline
        Ketterät menetelmät & agile, scrum, lean, xp \\ 
        Organisaatiomuutos  & transformation, transition, change, migration \\
        Suuri organisaatio  & enterprise, organization, (large \texttt{AND} scale) \\
        \hline
    \end{tabular}
	\caption{Hauissa käytetyt näkökulmat ja niitä vastaavat avainsanat}
	\label{table:hakulausekkeet}
\end{table}

--> Esihaut: IEEE, hakusanat --> Löytyi muutamia jotka eivät ole mukana muuten?

--> Viitteiden validiteetti (kuinka suuri luottamus)

--> Seulontamenetelmä: Otsikko ja abstrakti --> Kokoteksti

--> ''Data extraction form''

--> Data synthesis

% --------------------------------------------------------------------

\section{Tulokset}
\label{sec:tulokset}

--> ''Data synthesis'' \citep{Kitchenham2007}

--> Millä aikavälillä tutkimuksia on tehty

% --------------------------------------------------------------------

\section{Johtopäätökset}
\label{sec:johtopaatokset}




% --------------------------------------------------------------------

\section{Yhteenveto}
\label{sec:yhteenveto}




