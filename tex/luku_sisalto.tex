% --------------------------------------------------------------------

\textbf{Huomautuksia arvioijalle}

Symbolilla --> merkitsemäni kohdat ovat ideoita sisällöstä. Näiden olisi
kuitenkin tarkoitus kuvata lopullista sisältöä, eli niiden kuuluu olla
loogisessa järjestyksessä ja muutenkin olla järkeviä.

Olen merkinnyt kulmasulkeilla < > viitteet jotka toistaiseksi puuttuu lähteistä.

Olen pyrkinyt muotoilemaan uusiksi kaikki kohdat jossa oli käytetty ensimmäistä
personamuotoa. Sitä ei pitäisi enää esiintyä.

\newpage

\section{Johdanto}

Ohjelmistoalan kilpailutilanteen kiristyessä yritykset etsivät jatkuvasti tapoja
tehostaa toimintaansa. Ketterien menetelmien on todettu parantavan tehokkuutta
sekä laatua \citeppri{Livermore2008}, mikä nostaa ne houkuttelevaksi vaihtoehdoksi
tehostamista tavoitteleville yrityksille. Ketterien menetelmien käyttöönotto on
kuitenkin haastavaa suurissa yrityksissä \citeppri{Dyba2009}. Alun perin pieniin
projekteihin ja tiimeihin\footnote{Tässä työssä käytetään ohjelmistoalalla
vakiintunutta lainasanaa tiimi (engl. team) viitaten projektin työryhmään.}
suunnitellut mallit ovat osoittautuneet vaikeiksi soveltaa suuremmassa
mittakaavassa \citeppri{Boehm2005}.

Suuret yritykset toimivat usein perinteisten ohjelmistotuotannon mallien
mukaisesti. Nämä mallit pyrkivät optimoimaan toimintaa tarkalla suunnittelulla
ja prosessien määrittelyllä. Tämänlainen lähtökohta soveltuu kuitenkin huonosti
ohjelmistokehitykseen, sillä kehitysprojekteissa tulee lähes poikkeuksetta
tilanteita, joita on mahdotonta tai liian työlästä ennustaa \citeppri{Schwaber2002}.
Suurimpia ongelmia suunnitelmavetoisissa menetelmissä on vaatimusten muuttamisen
korkea hinta sekä myöhäinen palaute tuotteen laadusta \citeppri{Petersen2010}.
Pitkät julkaisuvälit, muutoksiin vastaamisen kalleus sekä etäisyys asiakkaista
heikentävät yritysten kilpailukykyä. Apua näihin ongelmiin toivotaan löytyvän
ketterien kehitysmallien soveltamisella.

Tämän työn tavoite on esittää nykyinen tutkimuksen tila ketterien
ohjelmistokehitysmenetelmien käyttöönotosta suurissa organisaatioissa,
tarkastellen erityisesti siihen liittyvää organisaatiomuutosta. Ketterien
menetelmien käyttöönotosta on olemassa tutkimuksia, mutta ne keskittyvät
enimmäkseen pieniin organisaatioihin tai yksittäisiin tiimeihin. Suuret
organisaatiot mukautuvat uusiin menetelmiin hitaammin, mikä voidaan olettaa
syyksi siihen, että laajaa tutkimusta suuren mittakaavan ketterästä muutoksesta
ei ole aikaisemmin tehty.

Tämä työ on toteutettu mukaillen järjestelmällisen kirjallisuustutkimuksen
muotoa, kartoittaen olemassa olevat tapaustutkimukset ja kokemusraportit.
Tutkimuskysymys johon tässä työssä vastataan on: Mitkä tekijät vaikuttavat
ketterän kehitysmallin organisaatiomuutoksen läpiviemiseen isossa
organisaatiossa? Työn tulos osoittaa, että ketterien kehitysmenetelmien
käyttöönotosta isoissa organisaatioissa on olemassa riittävästi ensisijaisia
tutkimuksia kirjallisuustutkimukseen.

Luvussa \ref{sec:tausta} esitellään aikaisempia tutkimuksia liittyen tämän työn
aihealueeseen, sekä todetaan että työ vastaa olemassa olevaan aukkoon
tutkimuksessa. Luvussa \ref{sec:menetelma} on esitelty järjestelmällisen
kirjallisuustutkimuksen menetelmä, sekä tapa jolla sitä on sovellettu tämän
kandidaatintyön kokeellisena osiona. Luvussa \ref{sec:tulokset} käsitellään
kirjallisuustutkimuksen tulokset. Lopuksi esitellään tuloksista tehdyt
johtopäätökset sekä työn yhteenveto.


% --------------------------------------------------------------------

\section{Työn taustat}
\label{sec:tausta}

Tämän työn tavoitteena on kuvata perusteita, haasteita ja menestyksen tekijöitä
ketterien kehitysmenetelmien käyttöönoton suurissa organisaatioissa. Näitä
\todo[size=\tiny]{Onko näin?}
tekijöitä tarkastellaan erityisesti organisaatiomuutoksen näkökulmasta.
Taustatietona työlle tässä luvussa esitellään edellisiä tutkimuksia jotka
liittyvät suuren mittakaavan ketterään kehitykseen, sekä ketterien menetelmien
käyttöönotton haasteisiin.
Luvun lopuksi perustellaan tämän työn merkitys suhteessa olemassa olevaan
tutkimuskeen.

\subsection{Ketterä ohjelmistokehitys}

Ketterä ohjelmistokehitys on joukko menetelmiä jotka on kehitetty vaihtoehdoksi
niin kutsutulle perinteiselle suunnitelmavetoiselle kehitykselle. Ketterien
menetelmien näkökulmasta perinteiset mallit pyrkivät vähentämään muutoksia
tuotteen kehityksessä, ja tulkitsevat muutoksia virheinä. Tuotteen
mukautumiskyky kehityskaaren aikana ja kyky vastata arvaamattoman maailman
haasteisiin ovat nykyään kuitenkin kriittisiä tekijöitä. Muutoksen omaksuminen
kehityksessä, korkean laadun tavoitteleminen ensi askelista alkaen, sekä
ihmiskeskeisyys ovat ketterien menetelmien keskeisiä teemoja.
\citeppri{Highsmith2001}

Ketterää kehitystä on sekä kritisoitu että puollettu, ja tutkimukset ovat
osoittaneet muutoksen omaksumisen sekä menestyksen että epäonnistumisen
tekijöinä \citeppri{Boehm2002}. On osoitettu että ketterät menetelmät ovat
parantaneet sekä kehittäjien että asiakkaiden tyytyväisyyttä, mutta toisaalta on
olemassa todisteita siitä, että ketterät menetelmät eivät välttämättä sovellu
suuriin hankkeisiin \citeppri{Dyba2009}. \citepri{Boehm2002} ehdottaakin että
organisaatioiden tulisi etsiä itselleen sopiva tasapaino suunnitelmavetoisten ja
ketterien menetelmien välillä.

Nykyään yleisiä ketterän kehityksen menetelmiä ovat Scrum ja Extreme Programming
(XP) \citeppri{Dyba2008}. Scrum on projekinhallintamenetelmä, joka painottaa
tiimikeskeistä toimintaa, ajallista rytmittämistä (time-boxing), jatkuvaa
edistymisen seurantaa, sekä asiakaskeskeisyyttä \citeppri{Schwaber2002}.
XP-menetelmä perustuu kokoelmaan käytäntöjä, joista keskeisimmät ovat jatkuva
palaute kehityksestä, testausvetoinen kehitys, asiakkaan jatkuva mukanaolo,
pariohjelmointi ja jatkuva yhdentäminen (engl. Continuous Integration
CI) \citeppri{Beck1999}.

\subsection{Ketterien menetelmien käyttöönoton haasteita}

Ketterät menetelmät eivät perustu yksittäisiin toimintatapoihin tai työkaluihin,
vaan kokonaisvaltaiseen toiminta- ja ajattelutavan muutokseen. Tämänlainen
muutos vaatii koko organisaatiokulttuurin muuttumista \citeppri{Misra2009}, mikä
on haastavampaa kuin pelkästään uusien työkalujen käytön opetteleminen. Muutosta
saattaa vaikeuttaa suunnitelmavetoiseen toimintamalliin juurtuneet ajattelutavat
ja asenteeet \citeppri{Mahanti2006,Nerur2005}. Ketterässä kehityksessä
organisaatiokulttuurin on oltava tiimi- ja asiakaskeskeinen, tiimeillä pitää
olla valinnanvapaus työkalujen ja menetelmien käytössä ja tiedon välittyminen
pitää mahdollistaa myös muissa muodoissa kuin kirjoitettuna dokumentaationa
\citeppri{Misra2009}.
Knowledge management -->

Kehittäjät ovat keskeisessä roolissa ketterissä malleissa. Organisaatiossa
joissa on vahva tausta suunnitelmavetoisesta kehityksestä saattaa kehittäjien
osaaminen olla kapea-alaista, mikä muodostuu ongelmaksi ketterien menetelminen
vaatiessa enemmän yleisiä taitoja \citeppri{Nerur2005}. Keskeinen ominaisuus
jota kehittäjien tulisi osoittaa ketterissä malleissa on itseohjautuvuus
\citeppri{Misra2009}. Ketterien menetelmien yhteydessä on suositeltu keskivertoa
parempien kehittäjen käyttämistä \citeppri{Boehm2002}. Pelkästään
huippukehittäjien käyttäminen ei ole välttämätöntä, mutta heikoimpien
suoriutujien mukanaolo on haitallista ketterässä kehityksessä
\citeppri{Nerur2005,Boehm2005}. Muutoksen myötä kehittäjien on omaksuttava uusia
työtapoja, mikä saattaa aiheuttaa muutosvastarintaa. Eräitä syitä
muutosvastarintaan voivat olla pelko omien taitojen riittämättömyydestä uudessa
ympäristössä tai kateuden tunne työtovereiden uusista tehtävistä
\citeppri{Boehm2005}.

Tiimien prosessien lisäksi myös liiketoiminnan prosesseja joudutaan sopeuttamaan
ketteriin menetelmiin. Tuotannon ja suunnittelun tulisi etääntyä
elinkaariajattelusta ja painottaa iteratiivista ja toiminnallisuuksiin
keskittyvää mallia \citeppri{Nerur2005}. Ongelmana on että perinteisissä
malleissa on annettu pitkän aikavälin suunnitelmia ja tehty sitoomuksia, mutta
ketterät menetelmät perustuvat nopeiden muutosten mahdollistamiseen. Tämän takia
organisaation johdon on hyväksyttävä muutosten tekeminen \citeppri{Misra2009}.
Perinteisesti liiketoimintaa on ojhattu pitkän tähtäimen suunnittelulla, mutta
ketterät menetelmät painottavat että merkityksellistä suunnittelua voi tehdä
vain lähitulevaisuuteen \citeppri{Boehm2005}. Johdon on myös hyväksyttävä että
kehitystyötä ei kannata pyrkiä hallitsemaan yksityiskohtaisella tasolla, vaan
sallittava kevyemmät raportointikäytännöt \citeppri{Cohn2003}.

Ketterät menetelmät vaativat asiakkaan vahvempaa läsnäoloa, mutta kaikki
asiakkaat eivät halua tai voi osallistua kehitykseen \citeppri{Nerur2005}.
Asiakkailta vaaditaan luottamusta kehitystiimiin, ja halua neuvotella
mahdollisista muutoksista \citeppri{Misra2009}. Myös sopimuskäytäntöjä pitää
muuttaa perustumaan toimivan ohjelmiston toimittamiseen sen sijaan että
keskitytään muodollisiin katselmointeihin tai dokumentaatioon
\citeppri{Boehm2005}.

Suurissa organisaatioissa ketterien menetelmien käyttöönotto on haastavampaa
kuin pienissä \citeppri{Livermore2008}. Tärkeä tekijä suurissa organisaatioissa
on se, että projektit ovat harvoin riippumattomia toisista. Riippuvuudet
kasvattavat tiimin tarvetta kommunikoida ulospäin. Suuri mittakaava saattaa
pakottaa ketteryyden vähentämistä esimerkiksi muodollisen dokumentaation
lisäämisellä \citeppri{Lindvall2004}. Ketterät menetelmät saattavat myös joutua
törmäyskurssille organisaation muiden toimintojen kanssa. Muun muassa
henkilöstöhallinnon käytännöt saattavat estää työntekijöitä ottamasta uusia ja
laajempia rooleja, mitä ketterä kehitys vaatii \citeppri{Boehm2005}.
Muutoksenhallintalautakunta (engl. change control board, CCB) saattaa
hankaloittaa refaktoroinnin ja jatkuvan yhdentämisen käyttöä
\citeppri{Lindvall2004}. Organisaation kehitystyöhön liittymättömiä toimintoja
pitää informoida muutoksesta ketterään kehitykseen, ja ketterää mallia pitää
sovittaa toimimaan yhteen niiden kanssa
\citeppri{Lindvall2004,Cohn2003,Boehm2005}. Liittyminen ympäröivään
organisaatioon saattaa tulla sillä kustannuksella, että on tuettava
lopputuotteen kannalta vähemmän arvoa lisääviä käytäntöjä, kuten dokumentaatiota
ja jäykempää suunnittelua \citeppri{Mahanti2006}.

\subsection{Ketterien menetelmien käyttöönotto}

Johdon puolto:
Kuten organisaatiomuutoksessa yleensä, myös ketterän kehityksen käyttöönotossa
on johdon myötämielisyyden raportoitu olevan keskeinen menestyksen tekijä
\citeppri{Livermore2008}.

Pilotointi:
--> Boehm: Pilotoinnin tulosten kanssa pitää olla varovainen

Koulutus ulkoisten resurssien käyttö:
--> \citepri{Livermore2008} esittää tutkimuksessaan että panostuksella
koulutukseen oli vahva yhteys onnistuneeseen muutokseen.
--> Boehm: counterin mythology through education (muutosvastarinta, etc.)

Muuta:
--> Menetelmien räätälöinti (Lindvall, )
--> Menetelmien tarkka valitseminen
--> Muutoksen ydinalueiden tunnistaminen ja niihin keskittyminen (Misra)

\subsection{Aikaisemmat tutkimukset <-- muuta otsikko}

--> Dybån ja Dingsøyrin tutkimuksen lisäksi tehtiin haku ketterien menetelmien
käyttöönottoa tai suuren mittakaavan soveltamista kuvaavista tutkimuksista.
--> löytyi 5 kappaletta aikaväliltä 2004-2009
--> Nämä ovat tämän tutkimuksen aihealueeseen vahvasti liittyviä
--> Alla on esitelty erityishuomioita kustakin tutkimuksesta

\citeppri{Dyba2008} ovat julkaisseet ainoan laajan ja järjestelmällisen
kirjallisuustutkimuksen joka yleisellä tasolla tarkastelee ketterää
ohjelmistokehitykstä. Tutkimuksen tuloksena todettiin muun muassa, että ketteriä
menetelmiä on vaikea soveltaa isommassa mittakaavassa kuin ''pienet tiimit''.
--> 7 tutkimusta 36 käsittelivät ketterän menetelmän käyttöönottoa.
--> Extreme Programming oli lähes ainoa menetelmä josta oli tehty tutkimusta.
--> Tutkimuksen tekijät suosittelivat suositun Scrum-menetelmän tutkimista.
\citeppri{Dyba2008}

\citeppri{Lindvall2004}:
--> Kvalitatiivista dataa neljästä organisaatiosta. \newline
--> Suurten organisaatioiden erityispiirre on tiimienvälisen koordinoinnin
tarve. Koordinoinnin takia tiimien on välttämättä tuotettava lisää
dokumentaatiota ja pidettävä kokouksia toisten tiimien kanssa, mikä vähentää
ketteryyttä. \newline
--> Ketterät menetelmät saattavat joutua törmäyskurssille organisaation olemassa
olevien toimintojen kuten muutoslautakunnan (engl. change control board) kanssa. \newline
--> Ketterien menetelmien sovittaminen olemassaoleviin prosesseihin
--> Organisatorilliset prosessit

\citeppri{Nerur2005}:
--> On ymmärrettävä että uuden menetelmän käyttöönotto aiheuttaa muutoksia koko
organisaation laajuisesti. \newline
--> Organisaatiokulttuuri on uudistettava, sillä vanhat toimintatavat ovat usein
syvälle juurtuneita. \newline
--> Toimintatapojen on muututtava ihmislähtöisiksi. \newline
--> Tuotantoprosessin on muututtava elinkaarimallista tukemaan
ominaisuuspohjaista (engl. feature-based) kehitystä. \newline
--> Uusiin työkaluihin on panostettava sekä hankintakustannuksia että
käyttöönoton vaatima työ.

\citeppri{Boehm2005}:
--> Konfliktit kehitysprosessissa: Kehitysprosessin muutokseen pitää
valmistautua huolella, ja prosessi pitää rakentaa alhaalta ylös. \newline
--> Liiketoimintaprosessien knofliktit: Liiketoiminnan johdon on hyväksyttävä
että tarkoilla aika- ja kustannusarvioilla ei ole sijaa ohjelmistokehityksessä.
Myös sopimuskäytännöt pitää uusia, mikä saattaa vaatia neuvotteluja asiakkaan
kanssa. \newline
--> Ihmislähtöisyys on kettrien menetelmien lähtökohta. Ihmisten välinen
interaktio on avaintekijä ketterässä kehityksessä. Tämä on huomioitava
johtamisen tyylissä, joka on perinteisissä organisaatioissa usein määräävä. On
myös oletettavaa että osa henkilöstöstä ei halua muuttaa toimintatapojaan.
Näiden haasteiden selvittäminen vaatii henkilöstön kouluttamista, sekä
ymmärrystä siitä miten ihmisten väliset interaktiot luovat sujuvaa
kommunikaatiota.

\citeppri{Livermore2008}:
--> Verkossa toteutettu kyselytutkimus, joka kartoitti mitkä tekijät korreloivat
ketterien menetelmien menestyneen käyttöönoton kanssa. \newline
--> Uudet menetelmät vaativat koulutusta -- koulutus oli merkittävä tekijä. \newline
--> Johdon tuki korreloi vahvasti \newline
--> Tarkan muutossuunnitelman laatiminen ei vaikuttanut olevan menestyksen tekijä \newline
--> Ulkopuolisen avun käyttäminen, esimerkiksi konsulttien, oli vahvasti
kytköksissä menestyksekkääseen muutokseen. \newline
--> Muutokset olivat onnistuneet pienissä organisaatioissa, kun taas suurilla
organisaatioilla oli ollut vaikeuksia. \newline
--> Johtopäätöksenä: pitää varata tarpeeksi resursseja muutosta varten.

\citeppri{Misra2009}:
--> Suuren mittakaavan kyselytutkimus. \newline
--> Määräävän ja määriteltyihin käytäntöihin perustuvan organisaatiokulttuurin
on muututtava sallimaan vapaus valita toimintatavat ja hajautettu johtaminen.
Organisaation on myös muututtava asiakaskeskeiseksi. \newline
gg--> Johtamistyylin on muututtava avoimemmaksi ja muutoksia sallivammaksi.
Johdon on hyväksyttävä riskien ja epävarmuuden olemassaolo, ja luotettava kehittäjiin.
Raportoinnin on perustuttava rehellisyyteen ja läpinäkyvyyteen, ja projektien
tila on esitettävä avoimesti. \newline 
--> Kehitysprosessin syklit on muutettava lyhyiksi ja kehityksessä on sallittava
muutoksia. \newline
--> Koko henkilöstöä on koulutettava ketterien mallien käytöstä ja periaatteista.
Erityisesti johdon kouluttaminen on tärkeää. \newline
--> Esitetyillä tekijöillä ei ollut keskeistä prioriteettia, jota tutkimuksessa
haettiin, vaan kaikki tekijät vaikuttavat yhtä tärkeiltä muutoksen kannalta

Mahanti, Aniket: Challenges in Enterprise Adoption of Agile Methods - A Survey

\subsection{Yhteenveto aikaisemmista tutkimuksista}

--> Tärkeimmät teemat yhteenvetona. Jos pystyy petaamaan pohjaa tulosten
keskeisimmistä havainnoista, niin hyvä.

--> Organisaatiokulttuurin muuttaminen \newline
--> Koulutus

% Tässä oli ennen ''työn motivointi'', eli tutkimusaukon määrittely
\subsection{Uuden tutkimuksen tarve}
Kuten yllä havattiin, ketteristä menetelmien soveltamisesta on julkaistu useita
aikaisempia tutkimuksia. Myös suuren mittakaavan ketterästä kehityksestä on
tutkimuksia, kuten myös ketterien menetelminen käyttöönottoon liittyvästä
organisaatiomuutoksesta. Suureen mittakaavaan tai organisaatiomuutokseen
liittyvät tutkimukset ovat enimmäkseen olleet primääritkimuksia, mukaan lukien
tapaustutkimuksia sekä kokemuskertomuksia. Laajaa kirjallisuustutkimiusta
keterän kehitysmallin käyttöönottoon liittyvään organisaatiomuutokseen ei ole
tehty. Näin ollen nykyiset julkaisut jättävät selkeän aukon yhteenvetävälle
kartoitukselle raportoiduista havainnoista suuren mittakaavan ketterässä
kehityksessä.

% --------------------------------------------------------------------

\section{Tutkimusmenetelmä}
\label{sec:menetelma}

Tässä luvussa esitellään työn tavoitteet sekä tutkimusmenetelmä jota on käytetty
niiden saavuttamiseksi. Ensiksi esitellään työn tutkimuskysymykset sekä rajaus.
Tämän jälkeen esitellään järjestelmällisen kirjallisuustukimuksen suorittamisen
periaatteet, ja kuvataan miten sitä on sovellettu tässä työssä. Tämän jälkeen
esitellään menetelmän vaiheiden suorittaminen tämän työn puitteissa.

\subsection{Tutkimuskysymykset}
Tutkimuskysymys kiteyttää työn tutkimuksellisen tavoitteen, ohjaten
tutkimusmenetelmän suunnittelua sekä tulosten analysointia. Tämän työn
tutkimuskysymys on: \textit{Mitkä tekijät vaikuttavat ketterän kehitysmallin
organisaatiomuutoksen läpiviemiseen isossa organisaatiossa?} Tutkimuskysymys on
edelleen jaettu kolmeen alla listattuun alakysymykseen.

\begin{itemize}
\item Miksi organisaatiomuutokseen ryhdytään?
\item Minkälaisia organisaatiomuutoksia on raportoitu?
\item Mitkä ovat muutosprosessin menestyksen ja ongelmien tekijät?
\end{itemize}

\subsection{Työn rajaus}
Kirjallisuustutkimuksessa huomioitavat lähteet on rajattu tutkimusten tyypin,
organisaation koon sekä muutosnäkökulman perusteella. Tutkimustyyppinä on
huomioitu ainoastaan toteutettuja tai käynnissä olevia organisaatiomuutoksia
käsitteleviä tapaustutkimuksia, monitapaustutkimuksia, sekä kokemusraportteja.
Tarkastelun ulkopuolelle on jätetty organisaatiomuutoksia tai ketteriä
menetelmiä yleisellä tasolla kuvaavat tutkimukset. Tutkitun organisaation tulee
olla riittävän iso, jotta voidaan katsoa ketterien menetelmien suuren
mittakaavan haasteiden tulevan esille \citeppri{Lindvall2004}. Tutkimukset jotka
käsittelevät vain yksittäisiä tiimejä on rajattu ulos. Organisaatiolla on oltava
määritelty aikaisempi toimintamalli siten, että tutkimuksessa esitetään
suunnitelmallinen muutos ketteriin menetelmiin. Muutoksen lopputilaa tai
organisaation uutta toimintamallia ei ole syytä rajata mihinkään tiettyyn
ketterään menetelmään, sillä käsitykset ketteristä menetelmistä saattavat
vaihdella tai muutos olla vielä kesken.

\subsection{Järjestelmällisen kirjallisuustutkimuksen menetelmä}

Tässä työssä käytetty tutkimusmenetelmä myötäilee Kitchenhamin esittämää mallia
ohjelmistotuotannon alan kirjallisuustutkimukseen. Valitun menetelmän tavoite on
muodostaa yhtenäinen kuva tämänhetkisestä tutkimuksesta työn aihealueella.
Ohjelmistotuotannon alan kirjallisuustutkimus suoritetaan pääasiallisesti
tekemällä hakuja elektronisiin tietokantoihin ja arvioimalla hakujen perusteella
löytynyttä aineistoa. Kirjallisuustutkimuks sisältää viisi pääasiallista
vaihetta: hakujen suunnittelu, aineiston seulonta, aineiston laadun arviointi,
tiedon poiminta sekä synteesin tekeminen. \citeppri{Kitchenham2007}

Järjestelmällisen kirjallisuustutkimukseen kuuluu lisäksi ulkopuolinen
katselmointi ja systemaattinen tutkimusprotokollan seuraaminen
\citeppri{Kitchenham2007}, mutta niiden soveltaminen ylittää kandidaatintyön
laajuuden.

\subsection{Hakujen suunnittelu}
Hakujen suunnittelulla tarkoitetaan käytettävien elektronisten aineistojen
valintaa ja hakulausekkeiden muodostamista. Elektronisina aineistoina käytettiin
seuraavia ohjelmistotuotannon alan julkaisuja hyvin kattavia tietokantoja:
IEEExplore <link>, ACM <link>, Scopus (ScienceDirect??) <link>, ProQuest <link>.
Näiden lisäksi suoritin haun <XP~Conference> arkistoon.

Ennen varsinaisten hakujen suorittamista suoritettiin esihakuja. Näiden
tarkoituksena oli kartoittaa hyviä avainsanoja varsinaisiin hakuihin. Esihauissa
käytettiin muun muassa termejä \textit{agile transformation} sekä \textit{large
scale agile}. Esihakujen perusteella löytyi useita aihealueeseen liittyviä
tutkimuksia.

Varsinaisissa hauissa käytettävät hakulausekkeet ovat boolen logiikalla
muodostettavia lausekkeita, jotka määräävät avainsanat joiden tulee esiintyä
hakutuloksissa. Hakulusekkeita varten määrittellään joukko näkökulmia jotka
edustavat tutkimuskysymysten aihepiiriä. Jokaista näkökulmaa kohden etsitään
yleisimmin käytettyjä avainsanoja sekä niiden synonyymejä. Lopuksi
hakulausekkeet muunnetaan boolen lausekkeiksi käyttämällä
\texttt{OR}-operattoria avainsanojen välillä, sekä \texttt{AND}-operaattoria
näkökulmien välillä. Taulukko~\ref{table:hakulausekkeet} esittää tässä työssä
käytetyt näkökulmat ja niitä vastaavat avainsanat.

\begin{table}[h]
    \begin{tabular}{|l|l|}
        \hline
        Näkökulma           & Avainsanat   \\ \hline
        Ketterät menetelmät & agile, scrum, lean, xp \\ 
        Organisaatiomuutos  & transformation, transition, change, migration \\
        Suuri organisaatio  & enterprise, organization, (large \texttt{AND} scale) \\
        \hline
    \end{tabular}
	\caption{Hauissa käytetyt näkökulmat ja niitä vastaavat avainsanat}
	\label{table:hakulausekkeet}
\end{table}

Kaikkia esihaun viitteitä ei löytynyt varsinaisessa haussa. Tämä johtuu siitä,
että osa tiivistelmistä on epäinformatiivisia, eivätkä sisällä avainsanoja.
Jotkut viitteet ovat myös otsikoitu epämääräisesti tai kekseliäästi, jonka takia
täsmällinen hakulausekke ei löydä niitä. Esihekujen löytämät viitteet olivat
kuitenkin selkeästi oleellisia, joten ne on liitetty varsinaisilla hauilla
kerättyyn aineistoon.

\subsection{Aineiston seulonta}

Kun suunnitellut haut on suoritettu on koottujen viitteiden joukosta seulottava
ne tutkimukset jotka ovat oleellisia kirjallisuustutkimuksen kannalta. Tämän
työn puitteissa seulonta suoritettiin kahdella tasolla: ensiksi otsikon
perusteella, ja sitten tiivistelmän perusteella. Otsikon perusteella hylättiin
vain selkeästi aihealueeseen kuulumattomat lähteet, sillä useat oleellisilta
vaikuttavat työt olivat otsikoitu epäselvästi. Otsikon perusteella valittiin 117
lähdettä. Tiivistelmän perusteella lähteet arvioitiin kolmesta näkökulmasta:
ketterä ohjelmistokehitys, suuri organisaatio tai monta tiimiä ja
toimintatapojen muutos. Tarkempaan tarkasteluun valittiin vain ne lähteet jotka
tiivistelmän perusteella käsitelivät kaikkia kolmea näkökulmaa. Tällä
menetelmällä tarkempaan tarkasteluun valittiin 31 lähdettä. Valituista lähteistä
yksi ei ollut saatavilla, jolloin tiedon poiminta suoritettiin lopulta 30
lähteelle.

\subsection{Tiedon poiminta}

Valitut ensisijaiset tutkimukset arvioidaan poimimalla niistä oleelliset tiedot.
Tiedon poimintaan käytetään tiedonkeruulomaketta, joka täytetään kutakin
tarkempaan tarkasteluun valittua lähdettä kohden. Tämän työn
tiedonkeruulomakkeessa käytettiin taulukossa \ref{table:dataform} listattuja
kenttiä.

\begin{table}[h]
    \begin{tabular}{| l | p{7.64cm} |}
        \hline
        Transformation mentioned in text (Y/N) &
        Large scale mentioned in text (Y/N) \\
        Is empirical case study (Y/N) &
        Is industry experience report (Y/N) \\
        Has listing of practices (Y/N) &
        Used research method (Y/N) \\
        Relevance to this review (1-5) & \\
        Objective of research (or publication) &
        Research method \\
        Author bias &
        Validity threats \\
        Organization size &
        Time of transformation \\
        Initial state of organization &
        Why was the change initiated \\
        How was the change conducted &
        What is agility? / Which agile paractices are used? \\
        Findings / lessons learned &
        Good practices validated or suggested by study \\
        Reported challenges &
        Satisfaction after transformation \\
        Effect on organization &
        Measurements as results (quantitative or other) \\
        Other notes &
        Notable references \\
        \hline
    \end{tabular}
    \caption{Työssä käytetyt tiedonkeruulomakkeen kentät. Kentät merkinnällä
    (Y/N) ovat kyllä/ei muotoisia kenttiä. Kentät merkinnällä (1-5) ovat
    pisteasteikon muodossa olevia kenttiä. Muut kentät ovat vapaamuotoisia
    tekstikenttiä.}
    \label{table:dataform}
\end{table}


% --------------------------------------------------------------------

\section{Tulokset}
\label{sec:tulokset}

Tässä luvussa esitellään synteesi ensisijaisista tutkimuksista ja raporteista
poimituista tiedoista. Lähteiden keruu ja tietojen poiminta on suoritettu
myötäillen luvussa \ref{sec:menetelma} esitetyä Kitchenhamin
kirjallisuustutkimuksen menetelmää \citeppri{Kitchenham2007}. Luvun alussa
esitellään yhteenveto tutkimuksista joita tarkasteltiin. Tämän jälkeen annetaan
yleiskatsaus tutkimuksissa kuvatuista ketteristä menetelmistä. Luvun loppuosa
käsittelee tutkimuskysymyksiin liittyvät oleellisimmat havainnot.

\subsection{Yhteenveto tutkimuksista}

Tutkitusta materiaalista 23 kappaletta oli teollisuuden kokemusraportteja, ja 7
kappaletta oli tapaustutkimuksia. Suurin osa julkaisuista oli
konferenssijulkaisuja, ja aineisto sisälsi vain kolme lehtiartikkelia. Tulosten
kannalta on tärkeä huomioida että suurin osa julkaisuista oli yritysten
työntekijoiden laatimia kokemusraportteja. Tämä seikkan valossa tuloksiin on
suhtauduttava varauksella. Erityisesti raportoituihin ongelmiin (tai niiden
puuttumiseen) ja positiivisiin tuloksiin on suhtauduttava varauksella.

Kuten kuva~\ref{fig:publications} näyttää, kaikki valitut tutkimukset olivat
julkaistu noin kymmenen vuoden sisällä. Tutkittu materiaali esittää nousun
ketterien menetelmien suosiossa vuoden 2007 kohdalla, mutta julkaisuiden
lukumäärä vaikuttaa pysyvän tasaisena. Huomattavaa on, että
organisaatiomuutoksia on raportoitu heti vuosituhannen vaihteen jälkeen, jolloin
ketterät menetelmät alkoivat tulla tunnetuksi. Tästä voidaan vetää johtopäätös,
että suuret yritykset ovat olleet kiinnostuneita ketterien menetelmien
soveltamisesta hyvin pian niiden esittelemisen jälkeen.

\begin{figure}[htb]
  \begin{center}
    \includegraphics[width=1\textwidth]{img/Publications}
    \caption{Julkaistut tutkimukset vuosittain}
    \label{fig:publications}
  \end{center}
\end{figure}

Suurimmassa osassa julkaisuja oli ilmoitettu ajankohta jolloin muutoksen ensi
askeleet oli otettu. Kuva~\ref{fig:start_year} esittää ilmoitetut muutosten
alkuajankohdat ryhmiteltynä vuosittain. Mediaani muutoksen aloitusajankohdasta
julkaisuvuoteen oli kolme vuotta. Tämä näkyy verrattavasti julkaisujen kasvun
nousussa vuonna 2007 ja aloitettujen muutoshankkeiden nousussa vuonna 2004.
Tästä voidaan päätellä että suuren mittakaavan muutokset ketteriin menetelmiin
ovat yleistyneet viime vuosikymmenen puolesta välistä alkaen. 

\begin{figure}[htb]
  \begin{center}
    \includegraphics[width=1\textwidth]{img/Transformation_start}
    \caption{Organisaatiomuutosten aloitusajankohdat vuosittain}
    \label{fig:start_year}
  \end{center}
\end{figure}

\subsection{Käytetyt ketterät menetelmät}

Kaikkissa tapauksissa organisaatio oli päätynyt jatkamaan ketterän mallin
käyttämistä. Muutoksen vaikutuksia oli tyypillisesti raportoitu 1 tai 2 vuoden
päähän sen aloittamisesta. Suurimmassa osasta tapauksia muutos katsottiin
edelleen jatkuvan raportin kirjoitusajankohtana.

Suurin osa tutkimuksta erotteli käytössä olevia ketteriä menetelmiä.
Taulukko~\ref{table:practices} esittää tutkimuksissa eniten raportoidut
menetelmät. Ylivoimaisesti suosituin ketterä menetelmä oli Scrum, jota
mainittiin käytettävän miltei kahdessa kolmesta organisaatiosta. Toiseksi
suosituin menetelmä oli Extreme Programming. Osassa organisaatioista myös
kevyt kehitysmalli (engl. lean development) oli yhdistetty ketteriin
menetelmiin.

Taulukosta~\ref{table:practices} ilmenee myös eri menetelmien yhdistely
huomattavana trendinä. Lähes kolmanneksessa raporteista mainittiin suoraan, että
menetelmiä oli yhdistelty. Menetelmien yhdistelemistä ja räätälöintiä oli
perusteltu pääasiallisesti kahdella tapaa. Osassa raporteista oli syyksi
mainittu se, että suuren organisaation erityispiirteet tekivät menetelmän
räätälöinnin välttämättömäksi. Toisaalta yritykset olivat halunneet panostaa
mahdollisimman hyvän menetelmän löytämiseen, jolloin useita menetelmiä oli
arvioitu ja niiden sopivimmat piirteet oli yhdistetty.

\begin{table}[h]
    \begin{tabular}{|l|l|}
        \hline
        Menetelmä       & N   \\ \hline
        Scrum           & 18 \\ 
        XP              & 7 \\
        Lean            & 5 \\
        Eri menetelmien yhdistely ja räätälöinti & 9  (epäsuorasti mainittuna useampia) \\
        \hline
    \end{tabular}
	\caption{Hauissa käytetyt näkökulmat ja niitä vastaavat avainsanat. Sarake N
	kuvaa kuinka monessa tutkimuksessa menetelmää mainittiin käytettävän.}
	\label{table:practices}
\end{table}

Edellä mainittujen ketterien mallien lisäksi oli yksittäisiä mainintoja
seuraavista menetelmistä: JAD (Joint Application Development),
ASSF\footnote{Qumer ja Henderson-Sellers [S9] kuvaavat ASSF-menetelmän}, Unified Process, sekä DSDM
(Dynamic Systems Development Method). Näiden menetelmien lisäksi mainittiin
erilaisia käytäntöjä mukaan lukien ajallinen rytmittäminen (engl. time-boxing),
testivetoinen kehitys, asiakkaan mukanaolo, testauksen automatisointi, sekä
jatkuva yhdentäminen.

\subsection{Tutkimuskysymyksiin liittyvät havainnot}

--> Johdantokappale: Saatiinko kysymyksiin vastauksia?

Huom: Kysymykset on ryhmitelty alaotsikoiksi kunnes keksin paremman järjestyksen!
 
\subsubsection{Miksi organisaatiomuutokseen ryhdytään?}

--> Vaatimus parannukseen (16 kpl)
--> Ongelmia havaittu nykyisessä mallissa (12 kpl)
--> Nopeampi TTM (10 kpl)

--> Kolmanneksessa tapauksista mainittiin että muutokseen ryhdyttiin erityisesti
yrityksen johdon aloitteesta.

--> Organisaatioiden lähtötila
--> Kolmannes tutkimuksista raportoi vesiputousmallin olevan käytössä
--> Lähes puolella tutkimuksista ilmoittivat tekevänsä pitkiä (vähimmillään
vuoden mittaisia) julkaisusyklejä.

\subsubsection{Minkälaisia organisaatiomuutoksia on raportoitu?}

--> Yleisin malli on askeleittainen käyttöönotto
--> Tutkimuksissa X ja Y mainittiin big-bang käyttöönotosta.

--> Pilotoinnin käyttö
--> Positiiviset kokemukset pilotoinnista
--> Pilotointi on kriittinen vaihe organisaatiomuutoksessa (tähän on olemassa
viite [S26] viitteissä)
--> Pilotointi nähtiin tapana hallita riskejä [S26]
--> Onnistuneet pilotointihankkeet merkitsivät usein käännekohtaa suurempaan
keterien menetelmien käyttöönottoon.

--> Konsulttien käyttö

--> Keskeisissä rooleissa olevien henkilöiden (oma-aloitteinen) kouluttautuminen.
--> Scrum y.m. sertifiointi
--> Coach communityn muodostuminen

--> Koulutuksen järjestäminen oli keskeinen tekijä monessa tapauksessa.
Ketterän koulutuksen tarve kasvoi sitä myötä kun yhä useampi tiimi valmistautui
ketterien menetelmien käyttöönottoon. <Joissakin tapauksissa piti hillitä
käyttöönottoa (?)>

--> Koulutuksen riittämättömyyden takia ketteriä menetelmiä ja niiden keskeisiä
periaatteitä ymmärrettiin väärin. <raportoitu tapauksissa X ja Y> Näissä
tapauksissa koulutuksen lisääminen paransi organisaation toimintaa.

--> Fyysisten tilojen järjestäminen nousi tärkeäksi seikaksi useassa tapauksessa.

--> Useassa tapauksessa mainittiin merkittävistä panostuksista jatkuvaan
testaukseen, sekä CI-teknologioihin.

--> Lopputuloksena oli ketterien menetelmien pysyvä käyttöönotto, paitsi [S5]
jossa organisaatio alkoi palautua käyttämään vanhaa menetelmää.
--> Organisaatioon tapahtuneista muutoksista raportoitiin pääasiallisesti
kahdella eri tavalla. Ensinnäkin joitakin vanhoihin toimintamalleihin
liittyneitä toimintoja lakkautettiin. Tähän liittyen oli myös raportoitu
muutoksia toiminnoissa joilla on rajapinta tuotekehitykseen, mutta ovat
osittain irrallisia kehityksestä. Esimerkiksi markkinointi jne. Toinen muutos
oli siirtyminen tiimikeskeiseen toimintaan.
--> Fyysinen tiimitila. --> Vähemmän thrashingia.

\subsubsection{Mitkä ovat muutosprosessin menestyksen ja ongelmien tekijät?}

Menestyksen ja ongelmien tekijät olivat osittain ongelmallisia tulkita, sillä
kaikki tutkimukset esittivät että ketterät menetelmät olivat jääneet käyttöön ja
lopputila oli positiivinen muutos. Negatiivisia tuloksia esittävien tutkimusten
puuttuessa ei ole mahdollista luokitella hyvään tai huonoon lopputulokseen
johtavia tekijöitä.


\vspace{1cm}
\textbf{Muuta}

--> Viitteiden validiteetti (kuinka suuri luottamus)

% --------------------------------------------------------------------

\section{Johtopäätökset ja yhteenveto}
\label{sec:johtopaatokset}

--> Muistutus työon tavoitteista (sidoksisuus johdantoon)

--> Yhteenveto tuloksista, ja tulosten merkitys
--> Suositus toimnepiteisiin??

--> Tulosten soveltuvuus
--> Tutkimuksen arviointi

--> Lisätutkimuskysymyksiä jatkoa varten: käyttöönoton laajuus /
oerganisaatiotyypit / mitkä menetelmät / enemmän perusteita miksi ?? 

--> Jatkotutkimukset
--> Pitää tehdä kirjallisuustutkimus seuraten huolellisesti jotakin
kijallisuustutkimuksen menetelmää. Jotta sain järjestelmällisen
kirjallisuustutkimuksen sovitettua kandidaatintyön laajuuteen jouduin
karsimaan useita oleellisia menetelmään kuuluvia muodollisuuksia.


