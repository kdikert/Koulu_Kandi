% --------------------------------------------------------------------

\section{Johdanto}

--> yritykset etsivät jatkuvasti tapoja parantaa toimintaansa (kiristyvässä kilpailutilanteessa). --> Tämä on nostanut ketterät ohjelmistomenetelmät mahdolliseksi ratkaisuksi toiminnan tehostamisessa.

--> Myös lean (Poppendick) on mahdollista --> miten se suhteutuu agileen

--> Tämän työn tavoitteenna on esittää semisystemaattisen kirjallisuuskatsauksen tulokset
--> 

\subsection{Työn rakenne}

Tämä työ on jaoteltu seuraavasti:
Seuraavassa luvussa esittelen tutkimuksent taustan ja määrittelen työn tavoitteet. Luvussa \ref{sec:menetelma} esittelen semisystemaattisen kirjallisuuskatsauksen menetelmän jota olen käyttänyt tutkimuksen suorittamiseen. Luku \ref{sec:tulokset} esittelee kirjallisuuskatsauksen tulokset. Lopuksi esittelen tuloksista tehdyt johtopäätökset sekä teen yhteenvedon työstä.


% --------------------------------------------------------------------

\section{Työn taustat ja tavoitteet}
\label{sec:tausta}

--> mitä ovat ketterät menetelmät


\subsection{Aikaisemmat tutkimukset}

--> Ei taida olla aikaisempia laajempia katsauksia
    * Miten perustellaan??
    
--> 

\subsection{Tutkimuskysymykset}


% --------------------------------------------------------------------

\section{Tutkimusmenetelmä}
\label{sec:menetelma}

\citep{refworks:148}



% --------------------------------------------------------------------

\section{Tulokset}
\label{sec:tulokset}

--> Millä ajalla tutkimuksia on tehty



% --------------------------------------------------------------------

\section{Johtopäätökset}
\label{sec:johtopaatokset}




% --------------------------------------------------------------------

\section{Yhteenveto}
\label{sec:yhteenveto}




