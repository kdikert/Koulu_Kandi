% --------------------------------------------------------------------

\textbf{Huomautuksia arvioijalle}

Symbolilla --> merkitsemäni kohdat ovat ideoita sisällöstä. Näiden olisi
kuitenkin tarkoitus kuvata lopullista sisältöä, eli niiden kuuluu olla
loogisessa järjestyksessä ja muutenkin olla järkeviä.

Olen merkinnyt kulmasulkeilla < > viitteet jotka toistaiseksi puuttuu läteistä.

\newpage

\section{Johdanto}

Ohjelmistoalan kilpailutilanteen kiristyessä yritykset etsivät jatkuvasti tapoja
tehostaa toimintaansa. Ketterien menetelmien on todettu parantavan tehokkuutta
sekä laatua \citep{Livermore2008}, mikä nostaa ne houkuttelevaksi vaihtoehdoksi
tehostamista tavoitteleville yrityksille. Ketterien menetelmien käyttöönotto on
kuitenkin haastavaa suurissa yrityksissä \citep{Dyba2009}. Alun perin
pieniin projekteihin ja tiimeihin\footnote{Selitys lainasanasta tiimi\ldots}
suunnitellut mallit ovat osoittautuneet vaikeiksi soveltaa suuremmassa
mittakaavassa \citep{Boehm2005}.

Suuret yritykset toimivat usein perinteisten ohjelmistotuotannon mallien
mukaisesti. Nämä mallit pyrkivät optimoimaan toimintaa tarkalla suunnittelulla
ja prosessien määrittelyllä. Tämänlainen lähtökohta soveltuu kuitenkin huonosti
ohjelmistosuunnitteluun, sillä kehitysprojekteissa tulee lähes poikkeuksetta
tilanteita joita on mahdotonta tai liian työlästä ennustaa \citep[ss. 97-100]{Schwaber2002}.
Suurimpia ongelmia suunnitelmavetoisissa menetelmissä on vaatimusten muuttamisen
korkea hinta sekä myöhäinen palaute tuotteen laadusta \citep{Petersen2010}.
Pitkät julkaisuvälit, muutoksiin vastaamisen kalleus, sekä etäisyys asiakkaista
heikentävät yritysten kilpailukykyä. Apua näihin ongelmiin toivotaan löytyvän
ketterien kehitysmallien soveltamisella.

Tämän työn tavoitte on kartoittaa nykyinen tutkimuksen tila ketterien
ohjelmistokehitysmenetelmien käyttöönotosta suurissa organisaatioissa. Ketterien
meneteilmien käyttöönotosta on olemassa tutkimuksia, mutta ne keskittyvät
enimmäkseen pieniin organisaatioihin tai yksittäisiin tiimeihin. Suuret
organisaatiot mukautuvat uusiin menetelmiin hitaammin, mikä voidaan olettaa
syyksi siihen että laajaa tutkimusta suuren mittakaavan ketterästä muutoksesta
ei ole aikaisemmin tehty. Tämä työ on toteutettu mukaillen jäejestelmällisen
kirjallisuustutkimuksen muotoa, kartoittaen olemassa olevia tapaustutkimuksia ja
kokemusraportteja. Työ osoittaa että ketterien kehitysmenetelmien käyttöönotosta
isoissa organisaatioissa on olemassa riittävästi ensisijaisia tutkimuksia
kirjallisuustutkimukseen.

Tämä työ on jaoteltu seuraavasti:
Seuraavassa luvussa esittelen tutkimuksen taustan ja määrittelen työn
tavoitteet. Luvussa \ref{sec:menetelma} esittelen semisystemaattisen
kirjallisuuskatsauksen menetelmän jota olen käyttänyt tutkimuksen
suorittamiseen. Luku \ref{sec:tulokset} käsittelee kirjallisuuskatsauksen
tulokset. Lopuksi esittelen tuloksista tehdyt johtopäätökset sekä teen
yhteenvedon työstä.


% --------------------------------------------------------------------

\section{Liittyvä tutkimus}
\label{sec:tausta}

Ketterästä kehityksestä suuressa mittakaavassa on tehty muutamia tutkimuksia
--> ''Motivation and background''??

--> Katso ''tätä lähellä olevia'' tutkimuksia, esim. 

--> Suunniteltu pienille tiimeille --> Ketterien ohjelmistokehitysmenetelmien on
kuitenkin väitetty olevan soveltumattomia suurille yrityksille. --> Vai onko
väitetty?? Pikemminkin vaan ehkä todettu että isot ovet erialisia kuin pienet
organisaatiot, ja sen takia ei käy tms.

--> Agiili on tehokkaampaa
  * <Empirical Studies on Quality in Agile Practices: A Systematic Literature Review; Panagiotis 2010>

--> Miksi ketteriin on siirrytty?

Vanhoja toimintamalleja:
-- waterfall
-- RUP
-- CMMi
--> Älä selitä näistä -- ehkä voi mainita, muttei selitä.
--> Mitä näistä voi sanoa? --> Onko niitä edelleen käytössä?

--> Jos vesiputousmalli on kerran kökkö, niin miksi isot organisaatiot ovat
ajautuneet käyttämään sitä (90-luvulla)?? --> Mistä viite\ldots?

--> Yhteenveto ongelmista?

--> Ihmisläheinen ''kaoottinen'' hallinta on välttämätöntä ohjelmistokehityksessä.
Suunnitalmavetoinen ja hallintaan perustuva ohjaus ei voi toimia, sillä
ohjelmistokehitys on aina lähempänä uuden tuotteen suunnittelua kuin
tehdaslinjastoa vastaavaa toistuvaa prosessia. <Schwaber \& Beedle 2002>


\subsection{Ketterät ohjelmistokenityksen menetelmät}

--> mitä ovat ketterät menetelmät

--> Mikä on organisaatiomuutos?

--> Ketterät kehitysmallit pyrkivät tuomaan parannuksia perinteisiin
ohjelmistokehityksen menetelmiin. --> Alla listattu mitä parannuksia voi olla

-- Kommunikaation parantaminen
  * Dokumentaation vähentäminen
  * Parempi yhteys asiakkaaseen
  * Kts. Bhalerao

-- Nopeampien muutosten salliminen
  * XP, Beck: Embracing change
  * Nopeammat syklit <Poole>

==> Mukautuimskyky?

-- Virheiden vähentäminen
  * Virheprosessin keventäminen (Korhonen??) -- ja edut?
  * Kts. Ileva
  * Kts.: Empirical Studies on Quality in Agile Practices: A Systematic Literature Review; Sfetsos

--> Yhteenveto ongelmista ketteryyden soveltamiseen ??
  * Lack of control? --> Pelko??

Verrattavia toimintamalleja:
--> Myös lean (Poppendick) on mahdollista
--> miten se suhteutuu agileen

\subsection{Aikaisemmat tutkimukset}

--> Ei ole aikaisempia laajempia katsauksia.
--> Tämä on selvinnyt liittyvien tutkimusten kartoittamisen myötä


% --------------------------------------------------------------------

\section{Tutkimusmenetelmä}
\label{sec:menetelma}

Tässä luvussa esittelen työn lähtökohdat tutkimuskysymysten sekä työn rajauksen
näkökulmista. Esittelen myös järjestelmällisen kirjallisuustutkimuksen
menetelmän, ja kuvaan miten olen soveltanut sitä tähän työhön.

\subsection{Tutkimuskysymykset}

Tämän työn tutkimuskysymyksenä on: \textit{Mitkä tekijät vaikuttavat ketterän
kehitysmallin organisaatiomuutoksen läpiviemiseen isossa organisaatiossa?}
Tutkimuskysymys on edelleen jaettu kolmeen alla listattuun alakysymykseen.

\begin{itemize}
\item Miksi organisaatiomuutokseen ryhdytään?
\item Minkälaisia organisaatiomuutoksia on raportoitu?
\item Mitkä ovat menestyksen ja ongelmien tekijät?
\end{itemize}

\subsection{Työn rajaus}

Rajaan kirjallisuustutkimuksessa huomioitavat ensisijaiset lähteet materiaalin
julkaisutyypin, organisaation koon sekä muutosnäkökulman perusteella.
Julkaisutyyppinä huomioin ainoastaan empiirisiä tutkimuksia, eli
tapaustutkimuksia, sekä kokemusraportteja. Tutkitun organisaation tulee olla
riittävän iso, jotta voidaan katsoa että ketterien menetelmien soveltamisen
erityishaasteet isoissa organisaatioissa tulevat esille \citep{Lindvall2004}.
Olen rajannut ulos tutkimukset jotka käsittelevät vain yksittäisiä tiimejä.
Organisaatiolla on myös oltava määritelty aikaisempi toimintamalli siten, että
tutkimuksessa esitetään suunnitelmallinen muutos ketteriin menetelmiin.
Muunnoksen lopputilaa tai organisaation uutta toimintamallia ei ole syytä rajata
mihinkään erityiseen ketterään menetelmään, sillä käsitykset ketteristä
menetelmistä saattavat vaihdella tai muunnos saattaa olla vielä kesken.

\subsection{Järjestelmällisen kirjallisuuskatsauksen menetelmä}

Tässä työssä käyttämäni tutkimusmenetelmä myötäilee Kitchenhamin esittämää
mallia ohjelmistotuotannon alan kirjallisuustutkimukseen. Valitun menetelmän
tavoite on muodostaa yhtenäinen kuva tämänhetkisestä tutkimuksesta työn
aihealueella. Kirjallisuustutkimuksen suorittaminen sisältää viisi pääasiallista
vaihetta: hakujen suunnittelu, aineiston seulonta, aineiston laadun arviointi,
tiedon poiminta sekä synteesin tekeminen. \citep{Kitchenham2007}

Järjestelmällisen kirjallisuustutkimukseen kuuluu lisäksi ulkopuolinen
katselmointi ja systemaattinen tutkimusprotokollan seuraaminen
\citep{Kitchenham2007}, mutta niiden soveltaminen ylittää tämän työn laajuuden.

\subsection{Hakujen suunnittelu}
Hakujen suunnittelulla tarkoitetaan käytettävien elektronisten aineistojen
valintaa ja hakulausekkeiden muodostamista. Elektronisina aineistoina käytin
seuraavia ohjelmistotuotannon alan julkaisuja hyvin kattavia tietokantoja:
IEEExplore <link>, ACM <link>, Scopus (ScienceDirect??) <link>, ProQuest <link>.
Näiden lisäksi suoritin haun <XP~Conference> arkistoon.

Hakulausekkeet ovat boolen logiikalla muodostettavia lausekkeita, jotka
määräävät avainsanat joiden tulee esiintyä hakutuloksissa. Hakulusekkeita varten
määrittellään joukko näkökulmia jotka edustavat tutkimuskysymysten aihepiiriä.
Jokaista näkökulmaa kohden etsitään yleisimmin käytettyjä avainsanoja sekä
niiden synonyymejä. Lopuksi hakulausekkeet muunnetaan boolen lausekkeiksi
käyttämällä \texttt{OR}-operattoria avainsanojen välillä, sekä
\texttt{AND}-operaattoria näkökulmien välillä.
Taulukko~\ref{table:hakulausekkeet} esittää tässä työssä käytetyt näkökulmat ja
niitä vastaavat avainsanat.

\begin{table}
    \begin{tabular}{|l|l|}
        \hline
        Näkökulma           & Avainsanat   \\ \hline
        Ketterät menetelmät & agile, scrum, lean, xp \\ 
        Organisaatiomuutos  & transformation, transition, change, migration \\
        Suuri organisaatio  & enterprise, organization, (large \texttt{AND} scale) \\
        \hline
    \end{tabular}
	\caption{Hauissa käytetyt näkökulmat ja niitä vastaavat avainsanat}
	\label{table:hakulausekkeet}
\end{table}

\subsection{Aineiston seulonta}

Kun suunnitellut haut on suoritettu on koottujen viitteiden joukosta sulottava
ne tutkimukset jotka ovat oleellisia kirjallisuustutkimuksen kannalta. Tämän
työn puitteissa suoritin seulonnan kahdella tasolla: ensiksi otsikon
perusteella, ja sitten tiivistelmän perusteella. Otsikon perusteella hylkäsin
vain selkeästi aihealueeseen kuulumattomat lähteet, sillä havaitsin että useat
oleellisilta vaikuttavat työt olivat otsikoitu epäselvästi. Otsikon perusteella
valitsin 100 lähdettä. Tiivistelmän perusteella arvioin lähteen kolmesta
näkökulmasta: ketterä ohjelmistokehitys, suuri organisaatio tai monta tiimiä ja
toimintatapojen muutos. Tarkempaan tarkasteluun valitsin vain ne lähteet jotka
tiivistelmän perusteella käsitelivät kaikkia kolmea näkökulmaa. Tällä
menetelmällä valitsin tarkempaan tarkasteluun 33 lähdettä.

\subsection{Tiedon poiminta}

Valitut ensisijaiset tutkimukset arvioidaan poimimalla niistä oleelliset tiedot.
Tiedon poimintaan käytetään tiedonkeruulomaketta, joka täytetään kutakin
tarkempaan tarkasteluun valittua lähdettä kohden. Tässä työssä käytin taulukossa
\ref{table:dataform} listattuja kenttiä.

\begin{table}
    \begin{tabular}{|l|l|}
        \hline
        Transformation mentioned in text (Y/N) &
        Large scale mentioned in text (Y/N) \\
        Is empirical case study (Y/N) &
        Is industry experience report (Y/N) \\
        Has listing of practices (Y/N) &
        Used research method (Y/N) \\
        Relevance to this review (1-5) & \\
        Objective of research (or publication) &
        Research method \\
        Author bias &
        Validity threats \\
        Organization size &
        Time of transformation \\
        Initial state of organization &
        Why was the change initiated \\
        How was the change conducted &
        What is agility? / Which agile paractices are used? \\
        Findings / lessons learned &
        Good practices validated or suggested by study \\
        Reported challenges &
        Satisfaction after transformation \\
        Effect on organization &
        Measurements as results (quantitative or other) \\
        Other notes &
        Notable references \\
        \hline
    \end{tabular}
    \caption{Työssä käytetyt tiedonkeruulomakkeen kentät. Kentät merkinnällä
    (Y/N) ovat kyllä/ei muotoisia kenttiä. Kentät merkinnällä (1-5) ovat
    pisteasteikon muodossa olevia kenttiä. Muut kentät ovat vapaamuotoisia
    tekstikenttiä.}
    \label{table:dataform}
\end{table}


\vspace{1cm}
\textbf{Muita juttuja / ideoita ??}

--> Esihaut: IEEE, hakusanat --> Löytyi muutamia jotka eivät ole mukana muuten?

--> Viitteiden validiteetti (kuinka suuri luottamus)

% --------------------------------------------------------------------

\section{Tulokset}
\label{sec:tulokset}

Tässä luvussa esittelen synteesin ensisijaisista tutkimuksista poimituista
tiedoista. Luvussa \ref{sec:menetelma} esittelin järjestelmällisen
kirjallisuustutkimuksen menetelmän ja listasin sen avulla lähdemateriaalista
poimitut tiedot \citep{Kitchenham2007}.

--> Millä aikavälillä tutkimuksia on tehty

--> Suosituimmat syyt lähteä muutokseen

--> Onko muutoksen toteuttamisessa joitain yleistettäviä piirteitä?

--> 

% --------------------------------------------------------------------

\section{Johtopäätökset ja yhteenveto}
\label{sec:johtopaatokset}

--> Muistutus työon tavoitteista (sidoksisuus johdantoon)

--> Yhteenveto tuloksista, ja tulosten merkitys
--> Suositus toimnepiteisiin??

--> Tulosten soveltuvuus
--> Tutkimuksen arviointi

--> Jatkotutkimukset


