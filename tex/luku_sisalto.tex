% --------------------------------------------------------------------

\section{Johdanto}

--> yritykset etsivät jatkuvasti tapoja parantaa toimintaansa (kiristyvässä
kilpailutilanteessa). --> Tämä on nostanut ketterät ohjelmistomenetelmät
mahdolliseksi ratkaisuksi toiminnan tehostamisessa.

--> Ketterien ohjelmistokehitysmenetelmien on kuitenkin väitetty olevan
soveltumattomia suurille yrityksille.

--> 

--> Viittaus edellisten töiden puutteeseen??

--> Tämän työn tavoitteenna on esittää semisystemaattisen kirjallisuuskatsauksen
tulokset -->

--> Yhteenveto tuloksista

(--> Yhteenveto johtopäätöksistä -- ei ollut suositeltu)



\subsection{Työn rakenne (jätä pois yksinäinen otsikko\ldots)}

Tämä työ on jaoteltu seuraavasti:
Seuraavassa luvussa esittelen tutkimuksen taustan ja määrittelen työn
tavoitteet. Luvussa \ref{sec:menetelma} esittelen semisystemaattisen
kirjallisuuskatsauksen menetelmän jota olen käyttänyt tutkimuksen
suorittamiseen. Luku \ref{sec:tulokset} esittelee kirjallisuuskatsauksen
tulokset. Lopuksi esittelen tuloksista tehdyt johtopäätökset sekä teen
yhteenvedon työstä.


% --------------------------------------------------------------------

\section{Työn taustat ja tavoitteet}
\label{sec:tausta}

\subsection{Perinteiset ohjelmistokenityksen menetelmät}

Vanhoja toimintamalleja:
-- waterfall
-- RUP
-- CMMi

--> Yhteenveto ongelmista?




\subsection{Ketterät ohjelmistokenityksen menetelmät}

--> mitä ovat ketterät menetelmät

Verrattavia toimintamalleja:
--> Myös lean (Poppendick) on mahdollista
--> miten se suhteutuu agileen


Oleellisia pointteja agilessa:

-- Kommunikaation parantaminen
  * Dokumentaation vähentäminen
  * Parempi yhteys asiakkaaseen
  * Kts. Bhalerao

-- Nopeampien muutosten salliminen
  * XP, Beck: Embracing change
  * Nopeammat syklit (Poole)
 
-- Virheiden vähentäminen
  * Kts. Ileva

--> Yhteenveto ongelmista ketteryyden soveltamiseen ??
  * Lack of control?


\subsection{Aikaisemmat tutkimukset}

--> Ei taida olla aikaisempia laajempia katsauksia
    * Miten perustellaan??
    
--> 

\subsection{Tutkimuskysymykset}

Tämän työn tutkimuskysymyksenä on: \textit{Mitkä tekijät vaikuttavat ketterän
kehitysmallin organisaatiomuutoksen läpiviemiseen isossa organisaatiossa?}
Tutkimuskysymys on edelleen jaettu kolmeen alla listattuun alakysymykseen.

\begin{itemize}
\item Miksi organisaatiomuutokseen ryhdytään?
\item Minkälaisia organisaatiomuutoksia on raportoitu?
\item Mitkä ovat menestyksen ja ongelmien tekijät?
\end{itemize}

\subsection{Työn rajaus}

--> Vain isot organisaatiot
--> Mikä on iso organisaatio? --> Vapaasit määriteltynä: sellainen jossa voidaan
olettaa että perinetisen projektisuunnittelun korvaaminen ketterillä
menetelmillä voi aiheuttaa koordinaatiovaikeuksia.

--> Organisaatiossa on oltava olemassa oleva toimintamalli

% --------------------------------------------------------------------

\section{Tutkimusmenetelmä}
\label{sec:menetelma}

\citep{refworks:148}

--> Esihaut: IEEE, hakusanat --> Löytyi muutamia jotka eivät ole mukana muuten?

--> Näkökulmat hakuun, vastaavat hakusanat ja boolean-lause: agile,
transformation, large scale

--> Käytetyt hakukoneet

--> Viitteiden validiteetti (kuinka suuri luottamus)

--> Seulontamenetelmä: Otsikko ja abstrakti --> Kokoteksti

--> ''Data extraction form''

% --------------------------------------------------------------------

\section{Tulokset}
\label{sec:tulokset}

--> Millä aikavälillä tutkimuksia on tehty

% --------------------------------------------------------------------

\section{Johtopäätökset}
\label{sec:johtopaatokset}




% --------------------------------------------------------------------

\section{Yhteenveto}
\label{sec:yhteenveto}




