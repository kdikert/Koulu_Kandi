% ---------------------------------------------------------------------
% -------------- PREAMBLE ---------------------------------------------
% ---------------------------------------------------------------------
\documentclass[12pt,a4paper,finnish,oneside]{article}
%\documentclass[12pt,a4paper,finnish,twoside]{article}
%\documentclass[12pt,a4paper,finnish,oneside,draft]{article}

\usepackage[utf8]{inputenc}

% Valitse 'output/font encoding':
%\usepackage[T1]{fontenc}      % korjaa ääkkösten tavutusta, bittikarttana
\usepackage{ae,aecompl}       % ed. lis. vektorigrafiikkana bittikartan sijasta

% Kieli- ja tavutuspaketit:
\usepackage[english,swedish,finnish]{babel}

% Kurssin omat asetukset aaltosci_t.sty:
\usepackage{aaltosci_t}

% Muita paketteja:
%\usepackage{alltt}
%\usepackage{amsmath}
\usepackage{calc}      % käytetään laskurien (counter) yhteydessä (tiedot.tex)
%\usepackage{eurosym}   % eurosymboli: \euro{}
\usepackage{url}       % \url{...}
\usepackage{listings}  % koodilistausten lisääminen
%\usepackage{algorithm} % algoritmien lisääminen kelluvina
%\usepackage{algorithmic} % algoritmilistaus
\usepackage{hyphenat}
\usepackage{supertabular,array}  % useampisivuinen taulukko

% Strikethrough
\usepackage[normalem]{ulem}


% Monta lähdeluoetteloa
\usepackage{multibib}
\newcites{pri}{Lähteet}

% Lähteiden listaus ilman lähdeluoetteloa
\usepackage{bibentry}


% Tavutus.
%\hyphenation{vää-rin me-ne-vi-en sa-no-jen tavu-raja-ehdo-tuk-set}
\hyphenation{me-ne-tel-män me-ne-tel-mis-tä tar-kas-te-luun}
\hyphenpenalty=10000   % rangaistaan tavutuksesta, 10000=ääretön
\tolerance=1000        % siedetään välejä riveillä
% titlesec-paketti auttaa, jos tämän mukana menee sekaisin

% Tekstiviitteiden ulkoasu.
% Pakettiin natbib.sty/aaltosci.bst liittyen katso esim. 
% http://merkel.zoneo.net/Latex/natbib.php
% jossa selitykset citep, citet, bibpunct, jne.
\bibpunct{(}{)}{;}{a}{,}{,}    % a = tekijä-vuosi (author-year)
%\bibpunct{[}{]}{;}{n}{,}{,}    % n = numero [1],[2] (numerical style)

% Rivivälin muuttaminen:
\linespread{1.24}\selectfont               % riviväli 1.5
%\linespread{1.24}\selectfont               % riviväli 1, kun kommentoit pois

% ---------------------------------------------------------------------
% -------------- DOCUMENT ---------------------------------------------
% ---------------------------------------------------------------------

\begin{document}

\selectlanguage{finnish}
\pagestyle{plain}
\pagenumbering{arabic}

\author{Kim Dikert}

\title{Ketterien kehitysmenetelmien käyttöönotto suurissa organisaatioissa: Kirjallisuustutkimus}


% Otsikko tiivistelmään. Jos lisäksi engl. tiivistelmä, niin viimeisin:
\TITLE{Ketterien kehitysmenetelmien käyttöönotto suurissa organisaatioissa: Kirjallisuustutkimus}
\ENTITLE{\LaTeX{} template for Bachelor thesis with a pretty long title %
line which continues ynd continues}

% Jos tarvetta 2-osaiselle otsikolle:
%\title{\LaTeX{}-pohja kandidaatintyölle \\[5mm] Pitkiä rivejä kokeilun vuoksi.}
%\TITLE{\LaTeX{}-pohja kandidaatintyölle. Pitkiä rivejä kokeilun vuoksi.}

% Ohjaajan laitos suomi/ruotsi ja tarvittaessa eng (tiivistelmän kieli/kielet)
\DEPT{Tietotekniikan laitos}
%\DEPT{Tietotekniikan laitos}
%\DEPT{Institutionen för datateknik}
%\ENDEPT{Department of Computer Science Engineering}

% Vuosi ja päivämäärä, jolloin työ on jätetty tarkistettavaksi.
\YEAR{2012}
\DATE{XX. marraskuuta 2012}
%\DATE{31. helmikuuta 2012}
%\DATE{Den 31 februari 2012}
\ENDATE{November XX, 2012}

% Kurssin vastuuopettaja ja työsi ohjaaja(t)
\SUPERVISOR{Ma professori Tomi Janhunen}
\INSTRUCTOR{TkT Maria Paasivaara}
%\INSTRUCTOR{Ohjaajantitteli Sinun Ohjaajasi, ToinenTitt Matti Meikäläinen}
% DI       // på svenska DI diplomingenjör
% TkL      // TkL teknologie licentiat
% TkT      // TkD teknologie doctor
% Dosentti Dos. // Doc. Docent
% Professori Prof. // Prof. Professor
% 
% Jos tiivistelmä englanniksi, niin:
\ENSUPERVISOR{Professor (pro tem) Tomi Janhunen}
\ENINSTRUCTOR{Maria Paasivaara, D.Sc.}
% M.Sc. (Tech)  // M.Sc. (Eng)
% Lic.Sc. (Tech)
% D.Sc. (Tech)   // FT filosofian tohtori, PhD Doctor of Philosophy
% Docent
% Professor

\PAAAINE{Ohjelmistotuotanto ja -liiketoiminta}
\CODE{T3003}

\KEYWORDS{agile, ketterä, ohjelmistokehitys, organisaatiomuutos}

\ENKEYWORDS{agile, software development, large scale, transformation}

% Kaikki sivut laskettuna nimiölehdestä lähdeluettelon tai 
% mahdollisten liitteiden loppuun. --> Miten liitteet (esim. "20 + 5")?
\PAGES{XX}

% ------------------------ Nimiölehti ---------------------------------

%\thispagestyle{empty}
\maketitle

% -------------- Tiivistelmä / abstract -------------------------------

\setcounter{page}{2}
% Tiivistelmät tehdään viimeiseksi. 
%
% Tiivistelmä kirjoitetaan käytetyllä kielellä (JOKO suomi TAI ruotsi)
% ja HALUTESSASI myös samansisältöisenä englanniksi.
%
% Avainsanojen lista pitää merkitä main.tex-tiedoston kohtaan \KEYWORDS.

\begin{fiabstract}

Ohjelmistoalan kilpailutilanteen kiristyessä yritykset etsivät jatkuvasti tapoja
tehostaa toimintaansa. Ketterien menetelmien on todettu parantavan tehokkuutta
sekä laatua, mikä nostaa ne houkuttelevaksi vaihtoehdoksi perinteisille
kehitysmenetelmille. Ketterien menetelmien käyttöönotto on kuitenkin haastavaa
suurissa yrityksissä, sillä ne on alun perin suunniteltu sovellettavaksi
pienissä projekteissa.

Tämän työn tavoitteena on selvittää mitkä tekijät vaikuttavat ketterän
kehitysmallin organisaatiomuutoksen läpiviemiseen suuressa organisaatiossa,
miten muutokset yleensä toteutetaan sekä miksi muutokseen ryhdytään. Hyödyntäen
järjestelmällisen kirjallisuustutkimuksen menetelmää löydettiin 30 ensisijaista
tutkimusta, jotka antoivat vastauksia näihin kysymyksiin. Tuloksissa esitellään
tyypillisiä muutoksen toteutustapoja, haasteita ja menestyksen tekijöitä.

Organisaatiomuutokseen ryhdyttiin kolmesta pääasiallisesta syystä, joita olivat
yleinen tarve tehostaa toimintaa, tiedostettujen prosessiongelmien poistaminen
ja tarve vastata markkinoiden muutoksiin nopeammin. Organisaatiomuutoksen
menestyksen tai epäonnistumisen keskeisimmäksi tekijäksi nousi muutoksen
johtaminen. Määrätietoinen johtaminen oli keskeisin menestyksen tekijä, ja
keskeisimmät ongelmat johtuivat vaikeuksista muodostaa yhtenäistä suuntaa
muutokselle kautta organisaation. Muita tärkeitä menestyksen tekijöitä oli
riittävä koulutuksen järjestäminen sekä yhteisöllisyyden luominen.
Pilotointi ja ulkopuolisten konsulttien käyttö oli tyypillistä muutoksissa.


%Tiivistelmän tyypillinen rakenne: 
%(1) aihe, tavoite ja rajaus 
%(heti alkuun, selkeästi ja napakasti, ei johdattelua);
%(2) aineisto ja menetelmät (erittäin lyhyesti);
%(3) tulokset (tälle enemmän painoarvoa); 
%(4) johtopäätökset (tälle enemmän painoarvoa).

%
%Tiivistelmätekstiä tähän (\languagename). Huomaa, että tiivistelmä tehdään %vasta kun koko työ on muuten kirjoitettu.
\end{fiabstract}

%\begin{svabstract}
%  Ett abstrakt hit 
%%(\languagename)
%\end{svabstract}

%\begin{enabstract}
% Here goes the abstract 
%%(\languagename)
%\end{enabstract}

\newpage

% -------------- Sisällysluettelo / TOC -------------------------------

\tableofcontents
\label{pages:prelude}
\clearpage

% -------------- Symboli- ja lyhenneluettelo -------------------------
%
%% -------------- Symbolit ja lyhenteet --------------


\addcontentsline{toc}{section}{Käytetyt symbolit ja lyhenteet}

\section*{Käytetyt lyhenteet ja termit}
%\section*{Abbreviations and Acronyms}

\begin{center}
\begin{tabular}{p{0.2\textwidth}p{0.65\textwidth}}
3GPP  & 3rd Generation Partnership Project; Kolmannen sukupolven  matkapuhelupalvelu \\ 
ESP & Encapsulating Security Payload; Yksi IPsec-tietoturvaprotokolla \\ 
\end{tabular}
\end{center}

\vspace{10mm}

Tähän voidaan listata kaikki työssä käytetyt lyhenteet. Lyhenteistä
annetaan selityksenä sekä alkukielinen termi kokonaisuudessaan
(esim. englanninkielinen lyhenne avattuna sanoiksi) että sama
suomeksi. Jos suoraa käännöstä ei ole tai sellaisesta on vaikea saada
sujuvaa, voi käännöksen sijaan antaa selityksen siitä, mitä kyseinen
käsite tarkoittaa. Jos lyhenteitä ei esiinny työssä paljon, ei tätä
osiota tarvita ollenkaan. Yleensä luettelo tehdään, kun lyhenteitä on
10--20 tai enemmän. Vaikka lyhenteet annettaisiinkin tässä
keskitetysti, ne pitää silti avata sekä suomeksi että alkukielellä
myös itse tekstissä, kun ne esiintyvät siellä ensi kertaa.  Käytetyt
lyhenteet -osion voi nimetä myös ``Käytetyt lyhenteet ja termit'', jos
luettelossa on sekä lyhenteitä että muuta käsitteenmäärittelyä.

\textbf{TIK.kand suositus: Lisää lyhenne- tai symbolisivu, kun se
  näyttää luontevalta ja järkevältä. (Käytä vasta kun lyhenteitä yli 10.)}


 
%\newpage

% ------------------------ Itse sisältö ------------------------------

% Muutetaan tarvittaessa ala- ja ylätunnisteet
%\pagestyle{headings}          % headeriin lisätietoja
%\pagestyle{fancyheadings}     % headeriin lisätietoja
%\pagestyle{plain}             % ei header, footer: sivunumero
%\pagenumbering{arabic}        % 1,2,3, samalla alustaa laskurin ykköseksi
%\thispagestyle{empty}         % pyydetty ensimmäinen tekstisivu tyhjäksi

% --------------------------------------------------------------------

\section{Johdanto}

--> yritykset etsivät jatkuvasti tapoja parantaa toimintaansa (kiristyvässä kilpailutilanteessa). --> Tämä on nostanut ketterät ohjelmistomenetelmät mahdolliseksi ratkaisuksi toiminnan tehostamisessa.

--> Myös lean (Poppendick) on mahdollista --> miten se suhteutuu agileen

--> Tämän työn tavoitteenna on esittää semisystemaattisen kirjallisuuskatsauksen tulokset
--> 

\subsection{Työn rakenne}

Tämä työ on jaoteltu seuraavasti:
Seuraavassa luvussa esittelen tutkimuksent taustan ja määrittelen työn tavoitteet. Luvussa \ref{sec:menetelma} esittelen semisystemaattisen kirjallisuuskatsauksen menetelmän jota olen käyttänyt tutkimuksen suorittamiseen. Luku \ref{sec:tulokset} esittelee kirjallisuuskatsauksen tulokset. Lopuksi esittelen tuloksista tehdyt johtopäätökset sekä teen yhteenvedon työstä.


% --------------------------------------------------------------------

\section{Työn taustat ja tavoitteet}
\label{sec:tausta}

--> mitä ovat ketterät menetelmät


\subsection{Aikaisemmat tutkimukset}

--> Ei taida olla aikaisempia laajempia katsauksia
    * Miten perustellaan??
    
--> 

\subsection{Tutkimuskysymykset}


% --------------------------------------------------------------------

\section{Tutkimusmenetelmä}
\label{sec:menetelma}

\citep{refworks:148}



% --------------------------------------------------------------------

\section{Tulokset}
\label{sec:tulokset}

--> Millä ajalla tutkimuksia on tehty



% --------------------------------------------------------------------

\section{Johtopäätökset}
\label{sec:johtopaatokset}




% --------------------------------------------------------------------

\section{Yhteenveto}
\label{sec:yhteenveto}





\label{pages:text}
\clearpage


% -------------- Lähdeluettelo / reference list ----------------------

% Viitetyylitiedosto aaltosci_t.bst; muokattu HY:n tktl-tyylistä.
\bibliographystyle{aaltosci_t}
\bibliographystylepri{aaltosci_t}

% Muutetaan otsikko "Kirjallisuutta" -> "Lähteet"
\renewcommand{\refname}{Lähteet}  % article-tyyppisen
%\renewcommand{\bibname}{Lähteet}  % jos olisi book, report-tyyppinen

\addcontentsline{toc}{section}{\refname}  % article
%\addcontentsline{toc}{chapter}{\bibname}  % book, report

\bibliographypri{lahteet}

\nobibliography{lahteet_s}

\clearpage

\section*{Ensisijaiset tutkimukset}
\addcontentsline{toc}{section}{Ensisijaiset tutkimukset}  


\begin{supertabular}{ l p{14.2cm} }
    {[S1]} & \bibentry{S1} \\ 
    {[S2]} & \bibentry{S2} \\ 
    {[S3]} & \bibentry{S3} \\ 
    {[S4]} & \bibentry{S4} \\ 
    {[S5]} & \bibentry{S5} \\ 
    {[S6]} & \bibentry{S6} \\ 
    {[S7]} & \bibentry{S7} \\ 
    {[S8]} & \bibentry{S8} \\ 
    {[S9]} & \bibentry{S9} \\ 
    {[S10]} & \bibentry{S10} \\ 
    {[S11]} & \bibentry{S11} \\ 
    {[S12]} & \bibentry{S12} \\ 
    {[S13]} & \bibentry{S13} \\ 
    {[S14]} & \bibentry{S14} \\ 
    {[S15]} & \bibentry{S15} \\ 
    {[S16]} & \bibentry{S16} \\ 
    {[S17]} & \bibentry{S17} \\ 
    {[S18]} & \bibentry{S18} \\ 
    {[S19]} & \bibentry{S19} \\ 
    {[S20]} & \bibentry{S20} \\ 
    {[S21]} & \bibentry{S21} \\ 
    {[S22]} & \bibentry{S22} \\ 
    {[S23]} & \bibentry{S23} \\ 
    {[S24]} & \bibentry{S24} \\ 
    {[S25]} & \bibentry{S25} \\ 
    {[S26]} & \bibentry{S26} \\ 
    {[S27]} & \bibentry{S27} \\ 
    {[S28]} & \bibentry{S28} \\ 
    {[S29]} & \bibentry{S29} \\ 
    {[S30]} & \bibentry{S30} \\ 
\end{supertabular}

%\bibliographysec{lahteet_s}


\label{pages:refs}
\clearpage         % erotetaan mahd. liitteet alkamaan uudelta sivulta


% -------------- Liitteet / Appendices --------------------------------
%
%\appendix
%\input{luku_liitteet}
%\label{pages:appendices}


% ---------------------------------------------------------------------

\end{document}
