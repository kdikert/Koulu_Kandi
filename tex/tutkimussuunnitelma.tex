

\documentclass[12pt,a4paper,finnish,oneside]{article}
%\documentclass[12pt,a4paper,finnish,twoside]{article}
%\documentclass[12pt,a4paper,finnish,oneside,draft]{article}

\usepackage[utf8]{inputenc}

% Valitse 'output/font encoding':
%\usepackage[T1]{fontenc}      % korjaa ääkkösten tavutusta, bittikarttana
\usepackage{ae,aecompl}       % ed. lis. vektorigrafiikkana bittikartan sijasta

% Kieli- ja tavutuspaketit:
\usepackage[english,swedish,finnish]{babel}

% Kurssin omat asetukset aaltosci_t.sty:
%\usepackage{aaltosci_t}

% Muita paketteja:
% \usepackage{amsmath}   % matematiikkaa
\usepackage{url}       % \url{...}

% Kappaleiden erottaminen ja sisennys
\parskip 1ex
\parindent 0pt
\evensidemargin 0mm
\oddsidemargin 0mm
\textwidth 159.2mm
\topmargin 0mm
\headheight 0mm
\headsep 0mm
\textheight 226.2mm

%\bibpunct{(}{)}{;}{a}{,}{,}    % a = tekijä-vuosi (author-year)

\pagestyle{plain}

% ---------------------------------------------------------------------

\begin{document}

\selectlanguage{finnish}
\pagestyle{plain}
\pagenumbering{arabic}

% Otsikkotiedot: muokkaa tähän omat tietosi

\title{TIK.kand tutkimussuunnitelma:\\[5mm]
Ketterien kehitysmenetelmien käyttöönotto suurissa organisaatioissa: Kirjallisuustutkimus}

\author{Kim Dikert, 54848S\\
Aalto-yliopisto,\\
\url{kdikert@cc.hut.fi}}

\date{\today}

\maketitle

% ---------------------------------------------------------------------

\vspace{10mm}

% MUOKKAA TÄHÄN. Jos tarvitset tähän viitteitä, käytä
% tässä dokumentissa numeroviitejärjestelmää komennolla \cite{kahva}.
%
% Paljon kandidaatintöitä ohjanneen Vesa Hirvisalon tarjoama 
% sabluuna. Kursivoidut osat \emph{...} ovat kurssin henkilökunnan
% lisäämiä. 

\textbf{Kandidaatintyön nimi:} Ketterien kehitysmenetelmien käyttöönotto suurissa organisaatioissa: Kirjallisuustutkimus

\textbf{Työn tekijä:} Kim Dikert

\textbf{Ohjaaja:} Maria Paasivaara


\section{Tiivistelmä tutkimuksesta}

Ketteriä kehitysmalleja on sovellettu jo yli vuosikymmenen ajan, mutta
enimmäkseen pienissä projekteissa. Myös suuret yritykset haluavat hyötyä
ketterien menetelmien eduista, mutta suuren organisaation toimintatapojen
muuttaminen on haasteellista. Tehtävänä on tehdä kirjallisuusselvitys tekijöistä
jotka vaikuttavat organisaatiomuutokseen joka tähtää ketterän kehitysmallin
käyttöönottoon.


\section{Tavoitteet ja näkökulmat}

Ohjaajan toimeksiannon perusteella teen työn myötäillen järjestelmällisen
kirjallisuuskatsauksen menetelmää. Tavoitteenani on siis kartoittaa tutkimuksen
nykyinen tila otsikon mukaisella aihealueella. Järjestelmällisen
kirjallisuustutkimuksen menetelmä voidaan katsoa kokeelliseksi työksi
kandidaatintyön puitteissa.

Kirjallisuustutkimuksen tuloksia tarkastellaan tutkimuskysymysten valossa:

\textit{Mitkä ovat tekijät jotka vaikuttavat agile-transformaation läpiviemiseen isossa
organisaatiossa?}

Alakysymys 1: Miksi transformaatioon ryhdytään? (olettaen että on jo ennestään olemassa speksattu toimintamalli)

Alakysymys 2: Miten transformaatio yleensä viedään läpi?

Alakysymys 3: Mitkä ovat menestyksen ja ongelmien tekijät?



\section{Tutkimusmateriaali}

Ensisijaisesti perustan tutkimuksen journal-artikkeleiden,
konferenssijulkaisujen, tutkimusraporttien sekä kokemuskertomusten varaan.
Käytän mahdollisesti muutamaa kirjaa materiaalina.

Valitsen materiaalin näin, sillä tavoittelen työssäni perusteltua tieteellistä
argumentointia. Käyttämieni viitteiden sisällön tulisi noudattaa tieteellistä
menetelmää.


\section{Tutkimusmenetelmät}

Sovellan tutkimusmenetelmänä ohjelmistoalan järjestelmällistä
kirjallisuustutkimusta kuten \cite{Kitchenham2007} esittää sen.
Kitchenham listaa myös alan oleelliset tietokannat joihin haut tehdään. Käytän
niitä tutkimuksessani.

Kirjallisuustutkimuksen lisäksi teen yleisiä hakuja työn aihealueeseen. Uskon
että järjestelmällisen kirjallisuustutkimuksen meentelmä antaa erinomaiset
eväät yleisten viitteiden hakuun.

\section{Haasteet}

Suurin haaste on se, että en koskaan ole tehnyt tällaista työtä ennen.
Akateeminen tutkimustyö on myös uusi asia. Suurin haaste on että en osaa
ennakoida mitä haasteita matkan varrella tulee. Erityisesti pelkään että teen
tutkimusmenetelmässä virheitä jotka laskevat lopputuloksen arvoa. Esimerkiksi,
saattaisin tehdä haut liian suppeasti, tai saattaisin kerätä epäoleellista
tietoa kerätystä aineistosta.

En usko että kirjoittaminen itsessään tai määrätietoisuuden puute olisi
haasteita. Kuvittelisin että nämä ovat yleisiä ongelmia, mutta koen kokeneena
opiskelijana hallitsevan nämä haasteet hyvin.


\section{Resurssit}

Työssä on kolme pääasiallista osiota: aihealueen aikaisempaan tutkimukseen
perehtyminen, järjestelmällisen kirjallisuuskatsauksen suorittaminen sekä
työn raportointi (itse kirjoittaminen). Aion suorittaa jokaisen työvaiheen
itse, mikä kurssilla ilmeisesti vaaditaankin.

Ohjaajani istuu samassa työhuoneessa kanssani. Hän on luvannut antaa neuvoja
jatkuvasti, ja luvannut että saan kysyä kysymyksiä päivittäin. Työn katselmointi
voidaan tehdä jatkuvalla mallilla -- ainakin alkuvaiheessa.

\section{Aikataulu}

Teen työtä niin paljon kuin mitä se vaatii. En pysty arvioimaan ajankäyttöä
oleellisesti paremmin, sillä minulla ei ole lainkaan kokemusta opinnäytetöiden
tai tutkimusten tekemisestä. Tämän takia joudun tarkkailemaan työn etenemistä
jatkuvasti, ja arvioimaan jos jokin vaihe vie liikaa aikaa. Olen vakuuttunut
että työn eri osa-alueita tehdään käytännössä lomittain.

Joudun palauttamaan työni jo marraskuun puolella, sillä minulla on työsopimus
joka edellyttää sen.

\begin{table}[h]
    \begin{tabular}{|l|l|}
	\hline
    Tavoite   & päivämäärä \\ \hline
    aiheeseen perehtyminen   & syyskuun puoliväli \\
    kirjallisuustutkimuksen materiaali  & lokakuun puoliväli \\ 
    kappaleet: johdanto, tausta ja menetelmä   & lokakuun loppu \\ 
    kappaleet: tulokset, yhteenveto   & marraskuun puoliväli \\ 
	\hline
	\end{tabular}
\end{table}


\section{Esittäminen}

Olen ajatellut esittää työn myötäillen ohjelmistotuotannon alan tietellisten
tutkimusten raporttien muotoa:

(1) Johdanto
(2) Aihealueen tausta, ja liittyvät tutkimukset
(3) Tutkimusmenetelmän ja tutkimustyön prosessin kuvaus
(4) Tulokset
(5) Yhteenveto


% ---------------------------------------------------------------------
%
% ÄLÄ MUUTA MITÄÄN TÄÄLTÄ LOPUSTA

% Tässä on käytetty siis numeroviittausjärjestelmää. 
% Toinen hyvin yleinen malli on nimi-vuosi-viittaus.

\bibliographystyle{aaltosci_t}
% \bibliographystyle{plainnat}
%\bibliographystyle{finplain}  % suomi
%\bibliographystyle{plain}    % englanti
% Lisää mm. http://amath.colorado.edu/documentation/LaTeX/reference/faq/bibstyles.pdf

% Muutetaan otsikko "Kirjallisuutta" -> "Lähteet"
\renewcommand{\refname}{Lähteet}  % article-tyyppisen

% Määritä bib-tiedoston nimi tähän (eli lahteet.bib ilman bib)
\bibliography{lahteet}

% ---------------------------------------------------------------------

\end{document}
